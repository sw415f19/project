\section{Introduction to Grammar}
This section will explain how the grammar for the syntax language will be generated and the reason why each part of the language was made as it is. 

\subsection{Context Free Grammar (ANTLR4)}
Context free grammar describes the syntax of the language. ANTLR4 which is the parser generator for this language has its own version of EBNF (Extended Backus–Naur form) it uses to implement the syntax.
The different annotation used in order to express EBNF can be seen in the table below.
\begin{center}
\begin{tabular}{ |c|c| } 
 \hline
 ? & Zero or one occurrence\\ 
 \hline
 | & One or the other \\ 
 \hline
 * & Zero or more occurrence \\ 
 \hline
  + & One or more occurrence \\ 
 \hline
\end{tabular}
\end{center}


\lstinputlisting[caption={EBNF for Ezuino}, label={code:ebnf},numbersep=5pt,language={[Plain]TeX}]{sections/language_design/syntax/Ezuino.g4}