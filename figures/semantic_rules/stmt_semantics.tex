\subsection{Statements}
\begin{table}[H]
    \centering
    \begin{longtable}[c] { r c }
    \begin{tabular}{@{}c@{}} 
    [ASS1] \\
    \newline \\
    \newline \\
    \newline \\
    \end{tabular}
  \begin{tabular}{@{}c@{}}   \( \langle x := a, sto, envl \rangle \Rightarrow (sto[l \mapsto v], envl) \)  \\ \( where \ envl  = (env_v, env_p) : envl' \)
  \\ \( and \ l = env_v x \)
  \\ \( and \ env_v, \ sto \vdash a \rightarrow_a v \)
  \end{tabular}
        
 \end{longtable}
    \caption{Small-step semantics for boolean assignment expressions}\label{sem:bool-ass}
\end{table}
        

\begin{longtable}[c] { r c }
  \centering
  [IF-ELSE-1] & \( 
    \dfrac { \langle e, envl, sto \rangle \Rightarrow_e \langle e', envl', sto' \rangle }
      { \langle if (e) \{S_1\} else \{S_2\}, envl, sto \rangle \Rightarrow_S \langle if (e') \{S_1\} else \{S_2\}, envl', sto' \rangle } \)
  \\
  & \\

  [IF-ELSE-2] & \( 
    \langle if (tt) \{S_1\} else \{S_2\}, envl, sto \rangle \Rightarrow_S \langle S_1, envl', sto' \rangle \)
  \\
  & \\

  [IF-ELSE-3] & \( 
    \langle if (ff) \{S_1\} else \{S_2\}, envl, sto \rangle \Rightarrow_S \langle S_2, envl', sto' \rangle \)
  \\
  & \\

  [IF-1] & \( 
    \dfrac { \langle e, envl, sto \rangle \Rightarrow_e \langle e', envl', sto' \rangle }
      { \langle if (e) \{S_1\}, envl, sto \rangle \Rightarrow_S \langle if (e') \{S_1\}, envl', sto' \rangle } \)
  \\
  & \\

  [IF-2] & \( 
    \langle if (tt) \{S_1\}, envl, sto \rangle \Rightarrow_S \langle S_1, envl', sto' \rangle \)
  \\
  & \\

  [IF-3] & \( 
    \langle if (ff) \{S_1\}, envl, sto \rangle \Rightarrow_S \langle skip, envl', sto' \rangle \)
  \\
  & \\

  [WHILE] & \( 
    \langle while (e) \{S\}, envl, sto \rangle \Rightarrow_S \) 
  \\
  & \(\langle if (e) \{S\ while (e') \{S\}\} else \{skip\}, envl', sto' \rangle \) 
  \\
  & \\

  [SKIP] & \( 
    \langle skip, envl, sto \rangle \Rightarrow_S \langle envl, sto \rangle \)
  \\
  & \\

  \caption{Small-step semantics for statements}
\end{longtable}

As previously mentioned, the function environment is a partial function from the name of the function to the 4-tuple of a sequence of commands, a set of parameters, a variable environment, and a function environment.
It should be noted, that the semantics for FUNC-DEC is written in way that resembles big-step semantics, as it is simpler and better conveys the meaning, rather than trying to declare 
\begin{table}[H]
    \centering
    \begin{longtable}[c] { r c }
        [FUNC-DEC] & \( \dfrac{env_v \vdash \langle D_F, env_F[f \mapsto (S, P, env_v, env_F)] \rangle \Rightarrow_{DF} env_F' } %lower half of dfrac below
        {env_v \vdash \langle func\ T \cup \epsilon\ f(T\ P_1, T\  P_2,...,T\ P_n)\{S\} D_F, env_F \rangle \Rightarrow_{DF} env_F'} \) \\
        \newline & where \(P = \{P_1, P_2,..., P_n\}\)\\
        \newline & and \(T \in \{int, bool, double\}\)\\
        \newline & and \(n \geq 0\)\\
    \end{longtable}
    \caption{Small-step semantics for function declaration}\label{sem:func}
\end{table}