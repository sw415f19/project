\section{Arduino}
The target group are users requiring a simple programming language for home automation as determined in \ref{targetedGroup}. As was previously concluded in chapter \ref{programmingparadigm}, the programming language should also be procedural in order to better accommodate users with less experience in programming. 
A programming language that fit the need of being procedural and is suited to implement software to physical devices like home automation is the Arduino programming language. Arduino is an open-source electronics platform based on easy-to-use hardware and software. Arduino uses boards that are able to read inputs - light on a sensor, a finger on a button, or a Twitter message - and turn it into an output - activating a motor, turning on an LED. Boards can be controlled by sending a set of instructions to the microcontroller on the board with the Arduino programming language. The Arduino programming language is a dialect of features from the programming languages C which is procedural imperative, C++, and the Arduino Software (IDE)\cite{audionocc}. \\ 

%The Arduino software is according to themselves, "easy-to-use for beginners, yet flexible enough for advanced users". It runs on Mac, Windows, and Linux. 

Another merit of creating a specialized version of the Arduino programming language is that the software is published as open source, and according to their official website, is available for extension by experienced programmers\cite{audionocc}.

%The environment also is written in Java and based on Processing and other open-source software.
%The language can be expanded through C++ libraries and runs on Windows, Mac OS X, and Linux. 







