\subsection{Problem scoping}
This section will describe which features we will not prioritize due to time limitations.\\
We will not prioritize list and structs. These data type are more complex than the other features to implement and while we do want to include both data types as we think they both serve important purposes we will only implement them if we have sufficient extra time.
We will not prioritize other control structures than the while loop. As mentioned previously the while loop enables everything the other control structures does and therefore the while loop will be prioritized.

\\
%\subsubsection{Scope of features}
%As there are a lot of IoT devices and other Arduino models, the project will be limited to be developed and tested on Arduino Uno.

%\subsubsection{Scope of tests}
%As it is essential that the compiler does not contain errors, testing will get focused to a greater extent of this project. Testing of the program, JUnit test should be done in the significant fields of code and functionality, together with running automated Continuous Integration (CI), which will run the tests in every build, together with a PR/Code Quality manager running as a static code analyser. Further testing, such as code coverage test, will not be considered in this development phase. \\


%As the scoping reach an end, we can start putting a problem definition together, where we can define the exact problem that we in which can solve with this programming language.