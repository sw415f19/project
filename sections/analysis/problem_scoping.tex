\section{Problem scoping}
This section will scope down the project, to which priorities are most vital for the project, to be called a success. \\
\\
The goal of the project is to create a programming language, which has the most essential syntax for the language to be use able in all cases. The language will be targeted for usage on an Arduino, and will have the most essential parts of the special syntax, the Arduino uses in its libraries. All the functionality of the Arduino library will not be implemented, as this is not a part of the project frame, and there is way to many special functions and syntax, that it will be scoped down to the most essential parts. \\
\\
The Ezuino programming language will only be code generated for two languages, Arduino C and Java bytecode. The syntax for each of these will be prioritized in a MoSCoW analysis, which will be reviewed in a later chapter. The Ezuino programming language will be made from the most prioritized features first, and going down to the less important features. As the programming language is very big to implement, the project will be scoped to implementing the vital parts at first, leaving some minor syntax priorities to may be discarded if time is critical. \\
\\
In a compiler, it is very important that it does not contain any errors, as it can be frustrating for the users programming their code in the language. As of this, it is vital to have testing, to a greater degree of the compiler. The testing of the project will be mainly focused on the technical part, with using JUnit testing and CI, to avoid any errors in the development process and a user test in the end to test the programming language. \\
\\ 
In the problem definition, the goal is to provide a programming language for people, with no or little programming knowledge. To provide a clear acceptance of this problem, multiple user testing should be made for well documenting this problem. In this project, the user testing will be scoped down to 1 user test, as the programming language itself, has the up most priority.

