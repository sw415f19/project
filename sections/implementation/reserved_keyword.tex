\section{Reserved keyword visitor}
As the name somewhat gives away, the reserved keyword visitor is the visitor which purpose is to check for reserved keywords.\\
The Ezuino language has a number of reserved keywords, which cannot be used in things such as variable and function declarations. To uphold this rule, the \texttt{ReservedKeywordsVisitor} class exists. The purpose of this class is to run throught the AST and verify than none of the reserved keywords are used for any declarations.
\\\\
The \texttt{ReservedKeywordsVisitor} extends from the abstract class \texttt{AstVisitor}, so that it can visit each node in the AST. As the purpose of this visitor is to check that no declaration uses a reserved keyword, the main functionality of the visitor, is to compare each variable and function declaration's Id with a list of reserved keywords. The Id of a variable and function declaration is the name used for that declaration.
\\\\
Listing \ref{code:RKW:keywordList} shows the constructor for the reserved keyword visitor, as well as the HashMap and HashSet that the keywords are stored in. The reason for having two sets to store reversed keywords, comes from the Ezuino specified keywords and Java reserved keywords.\\
Ezuino specified keywords are stored in the HashMap named \texttt{reservedKeywords}. These are stored with the upper case version of the keywords.\\
Java reserved keywords are keywords which are used in the Java language. As such the most common Java keywords, which are not already checked in Ezuino keywords, are stored in the HashSet named \texttt{compatibilityKeywords}.
\\\\

\ref{code:RKW:keywordCheck}

\begin{lstlisting}[caption={Constructor of the reserved keyword visitor where keywords are stored in a HashMap and HashSet.}, label={code:RKW:keywordList}]
private final Map<String, String> reservedKeywords = new HashMap<String, String>();
private final Set<String> compatibilityKeywords = new HashSet<String>();
private final ErrorHandler errorHandler;

public ReservedKeywordsVisitor(ErrorHandler errorHandler) {
    this.errorHandler = errorHandler;
    List<String> stringList = (Arrays.asList("return", "AND", "OR", "true", "false", "int", "double", "boolean", "string", "while", "if", "else", "Print", "DigitalWrite", "DigitalRead", "AnalogWrite", "AnalogRead", "Delay", "DelayMicro", "PinMode", "SerialBegin", "SerialEnd", "Setup", "Loop"));
    for (String word : stringList) {
        reservedKeywords.put(word.toUpperCase(), word);
    }
    compatibilityKeywords.addAll(Arrays.asList("goto", "Double", "float", "Integer", "for", "ArrayList", "List", "switch", "Collection", "main"));
}
\end{lstlisting}
\begin{lstlisting}[caption={Private functions in the reserved keyword visitor for comparing a given Id with reserved keywords.}, label={code:RKW:keywordCheck}]
private void checkReservedKeywords(String Id) {
    String key = Id.toUpperCase();
    if (reservedKeywords.containsKey(key) && (!reservedKeywords.get(key).equals(Id))) {
        errorHandler.reservedKeyword(Id, reservedKeywords.get(key));
    }
}

private void checkCompatibilityKeywords(String Id) {
    if (compatibilityKeywords.contains(Id)) {
        errorHandler.compatibilityKeyword(Id);
    }
}
\end{lstlisting}