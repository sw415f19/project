\chapter{Discussion}
In this chapter, we will discuss the aspects of the development of the Ezuino programming language. We will review the intended goal of the programming language and reflect whenever the final product reflects the goals, which were set for the programming language in the beginning.   
\\\\
In the problem definition, the following got defined:  
\\\\
\textit{In a world of increasing microchip automation, programming becomes a basic mainstream skill. To assess this, a programming language should be designed, which can be used for people, with little or no prior programming knowledge. } 
\\\\
The problem definition leans closely towards that the final programming language should be easily used for new or beginners. The syntax was designed within the language design criteria to be readable, writable and reliable. Using these criteria, we were able to choose a minimal syntax, allowing for the core functionality for an Arduino. Before starting to write the syntax, we are looking at the core programming language an Arduino is running, C with Arduino libraries. We then took the main features from the C programming language and found multiple alternatives of which syntax is best for each feature. We decided to keep the language very minimal, so most datatypes and control flow functionality is limited to the users being able to learn it fast, and don’t get overwhelmed by the features available. In the beginning, we had a general idea of how the program syntax should look like, from when we as developers started to program. After setting up this syntax, we had a user test with a third-party test person, who fitted the end goal group well, as they did not have any previous programming experience. The feedback received by this was surprising well for most of the syntax, other than naming some of the attributes and strong typing. The test person managed to create a small program, using the help sheet provided, which is a very good early reflection towards the program definition, where it should be used or learned by the target group.  
\\\\
During the development of the Ezuino programming language, reserved keywords was a greater challenge as well. The Java programming language, which is the language the programming language got developed in, had a lot of reserved keywords, which must be handled in the grammar files, and as well inside the compiler as well. As the end goal was towards the Arduino, which is using a library for most of its special syntax, the Ezuino programming language should know about these “special syntax” as well. Arduino uses the C programming language, with a special library designed to make development and usage of the Arduino easier for the users. This library was large, and most of the library content wouldn't fit the description of the problem definition, where the language must be easy to use and learn. It was decided that some of the features, which is vital for the Arduino to run, and the features with the core functionality and makes the most sense for the users, got added as well. In this consideration, we had to reflect whenever the input of the function like pinMode, which in the Arduino specification accepts both the HIGH / LOW keywords and using numbers which HIGH and LOW are defined in the library as 0 and 1. The Booleans in the Ezuino language is already defined as true and false, where the 0 and 1 are hidden for the user, so the same should be considered here, where we force the user to use the HIGH / LOW syntax for these special functions. \\\\
Scopes in the Ezuino programming language was thought to be the same, as in the C programming language, as this is the core language used for the Arduino. The reason for this, is that that the user in the user test could understand how it works, and it builds a form of hierarchy on each scope, meaning that you can access what has been added on a higher level, but not in the future and not in closed scoping. (MB BS???)  
\\\\
In the Ezuino programming language, it is vital to declare a variable before doing statements. This has been ruled in the ANTLR grammar file. The idea behind this, is to “create a contract” with the users, that they must do it consistently, both for the readability, but also to keep track of where and what variables have been declared where. There has been a lot of discussions whenever it is good to force the users to write more lines of syntax than necessary, however, it may be changed in the future if more user feedback can be assessed, even though new users most likely would find it normal, as it is most likely their first programming language that they have used.  
\\\\
Missing: types and tests, then done 

