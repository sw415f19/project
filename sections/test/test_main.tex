\chapter{Test}
In this chapter, we go through the tests, which got made during the development of the Ezuino programming language. At the beginning of the project, the only tests which could be done were using the language made by the ANTLR grammar file and testing different dummy programs, and test whenever ANLTR report any ambiguity or syntax error within the code entered. Multiple programs got written, to ensure that we got around every syntax and scenario possible, to avoid any future error, once we’ve started building the Concrete Syntax Tree (CST) and the Abstract Syntax Tree (AST).  One of the programs can be reviewed in listing \ref{ex01} and \ref{ex02}.
\\\\
\section{JUnit Testing}
Once the development of the compiler in Java has started, JUnit testing has been used during the remaining process. This section will go through some of the JUnit test, which has been written during the development process. The tests have been categorized into three sections, the tests for the concrete syntax tree, the abstract syntax tree and the lexer and the parser CST, AST, and Lexer/Parser test.
Starting from the beginning of development, we look at some of the Lexer and Parser tests, which is the classes ANTLR 4 has generated and using the ANTLR maven package, can use. In these test classes, we use the Error Handler, which was explained in \ref{error-handling-chap}, to check whenever there is an error, during the lexing/parsing of a small code snippet. If the error handler finds an error, depending on the assert, the test will either fail or pass.\\
Listing \ref{test1} shows a method, which is used to initialize the error handler, lexer and parser with the code, as inputted by a string.
\begin{lstlisting}[caption={Private function to run a test program}, label={test1}]
private ErrorHandler parseProgram(String program) throws IOException {
    CharStream stream = CharStreams.fromString(program);

    ErrorHandler errorHandler = new ErrorHandler();
    ErrorListener el = new ErrorListener(errorHandler);

    EzuinoLexer lexer = new EzuinoLexer(stream);
    CommonTokenStream tokens = new CommonTokenStream(lexer);
    EzuinoParser parser = new EzuinoParser(tokens);

    lexer.removeErrorListeners();
    lexer.addErrorListener(el);
    parser.removeErrorListeners();
    parser.addErrorListener(el);
    parser.start();

    return errorHandler;
}
\end{lstlisting}
\noindent\newline

On listing \ref{test2}, we have the first test, which a simple integer declaration, where we declare the variable a. We assert that the errorHandler should not have any errors, as this is the correct syntax in this case – using the hasErrors() method, which is a part of the error handler class, and will return an Boolean value.
\begin{lstlisting}[caption={Simple declaration test}, label={test2}]
$$@Test
public void testDcl() throws IOException {
    ErrorHandler errorHandler = parseProgram("int a");
    assertFalse(errorHandler.hasErrors());
}
\end{lstlisting}
\noindent\newline

Listing \ref{test3} shows the test of the expression (1 * 10), where we test if both the right and left side nodes of the MultiplicativeExprNode are of type Integers. If they aren’t, we can already throw an exception. Next, we’re checking if the values of the left and right nodes are the ones that we’ve entered in our expression. In this case, we assert that the left node has the value of 1, and the right node has the value of 10. We also check that the operator in this expression is a multiplicative operator.\\
The code generation tests for Java bytecode and C, has been tested by taking the Ezuino program, and comparing it to an expected output of the C programming language. If the output is correct, the tests are successful.
\begin{lstlisting}[caption={Simple multiplication expression test}, label={test3}]
$$@Test
public void simpleMultiplicativeExprTest() throws IOException {
    EzuinoParser ep = createParser("1 * 10");
    MultiplicativeExprNode topNode = (MultiplicativeExprNode) ep.expr().accept(visitor);

    assertTrue(topNode.getLeftNode() instanceof IntegerLiteral);
    assertTrue(topNode.getRightNode() instanceof IntegerLiteral);
    assertEquals("1", ((IntegerLiteral)topNode.getLeftNode()).getVal());
    assertEquals("10", ((IntegerLiteral)topNode.getRightNode()).getVal());
    assertEquals("*", topNode.getOperator());
}
\end{lstlisting}
\noindent\newline

\section{Continuous Integration}
During the development of the programming language, more tests has been added in the development process. The final number of tests, is around 500, in multiple classes. This is a significant number of tests to run each time an AST or node has been changed, so Continuous Integration (CI), has been deployed, to automate each test and compile the software after each Git Commit (GC) or Pull Request (PR).  The CI used for this purpose was Travis CI, which is a free CI service for public git repositories. \\
\begin{figure}[H]
\centering
\frame{\includegraphics[scale=0.8]{figures/newfuk.png}}
\caption{Screenshot from TravisCI website for Ezuino project}
\label{testa}
\end{figure}
On figure \ref{testa}, a Travis build has run on a GC, which has recently been committed. In this result, we can see that the 475 tests have been run successfully, and the program ran without any error. If any of the tests or main could not run, Travis would have thrown an error. \\
\\
This chapter has been going through a few of the around 500 tests made in this project and given an insight in the tests process which has been made during the development of the Ezuino programming language. It is very essential to avoid errors in the compiler in a programming language, so testing of features has been prioritized, however, there are still testing forms like code coverage and whitebox (branch testing), which has not been done at this time. If the Ezuino programming language were to get a commercial release, these tests are essential to provide a large sum of error or failure correction, as a programming language is very abstract. 
\section{User Test}
In this chapter, we present the feedback we were given from Ezuino. The test person was a student in psychology second semester. The testperson has never programmed before but was interested in participating in the test. The test was conducted by giving the test person a guide that both covered basic programming terms, the syntax of Ezuino with code examples, and a set of assignments. The guide can be seen in appendix \ref{user_guide_appendix}. The test person was taking the test together with a test moderator, in order to ensure the participant did not get completely stuck at a task either due to being completely new to programming or due to the guide being unclear. After the test, the participant was asked to give feedback on the programming language and to review key points about Ezuino where it differed from C. The screen and communication between the test moderator and the test person was recorded for internal review at a later time.
We got the following feedback
\begin{itemize}
 \item “AND” and “OR” was preferable to $\&\&$ and $\|\|$ and boolean true/false was preferred to 1 and 0.
 \item The data types could be less technical named, so it is easier to understand without a background. An example from the test person was that string could be called “words”. We asked if “text” would also be easy to understand as we felt it was more precise for more use cases of strings, and the test person said it was just as good. It could be observed from the test person that they fairly fast forgot was the meaning of int, double and boolean was after having read the explanation. While this could simply be due to having to comprehend many programming terms in a short time span, it could also be due to the name of the data types being less intuitive, making them harder to remember the meaning of.
 \item Strong typing also came as a surprise to the test person. Attempts of assigning a double variable to an integer("32"), was an example of this.
 \item We showed the test person the list we had intended to implement as versions of arrays that did not need malloc and free and functioned like ArrayList in Java but without being object oriented. The test person did prefer it but also suggested a method to find the index of a value in a string list in case they had a hard time keeping track of what index each value was saved. Interestingly enough the index of a value is often easier to remember in arrays compared to list when looking at the source code since you have to use the array index when you assign entries in an array. This suggests that having both arrays and a structure similar to list would be good since they both have different advantages for people new to programming. In both arrays and list the test person would prefer that entries in the array would start from index 1 instead of 0.
 \item According to the test person “:=” and “=” made sense compared to C, “:=” reminding the test person of assignment in a math program previously used in her study.
 \item During the test there were a few times the test person forgot declaration before assignment. This could be interpreted as a declaration before usage being unintuitive to new programmers.
\end{itemize}

Besides those points the test person also gave some feedback on our current error messages, making us aware that both the error messages from type mismatch and syntax error had some room for improvement.
This feedback confirms that most of the syntax we differ from C makes sense from a user perspective of our target group. It also gives feedback to several ways we could change the syntax, and the type checking to make it better suit new programmers needs.
