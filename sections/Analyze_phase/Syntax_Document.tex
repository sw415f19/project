In this chapter the syntax of the program will be explained and the reasons for them. Many of these syntaxes have been taking from the imperative language C, but also the object-oriented language Java. 

Here is a list of the Syntax %Insert listen

All the conditional operators have been spelled with capital letters. The reason for this is. so the conditional operators can be distinguised from the conditions and variables.

The next part are the beginning and end of the scope those will also be in capital letters because, an average user would easily if he is not aware that he is missing a { or } some where in the code. So that is why the begnning of the scope will be called BEGIN and the end of the scope will be called END

The if and else conditions will stay the same as they are in both C and Java because it is not complicated.

All the mathematic operations will be the same as they are, since no confusions can be made when using those operators.

When a variable needs to get its value the assignment := will be used to showcase the variable's value. The numerical comparison operators will use something similar with the  larger/less or equal operation, which looks like this >= or <=. The larger or less will always be on the left side to ensure no confusion for the user, when using them.

The not expression !, will stay the same as it is, just like Java and C

To comment in the language the user must write after a # to make comment. The # will only make a comment for a single line so if the users wanted to write a comment that is 2 lines long the user must do it like this.
#Hello
#world

To print a string or something else one must use the printf function, which is just like C