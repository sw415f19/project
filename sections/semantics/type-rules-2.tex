\subsection{Blocks}
Blocks can be referred to as the body of if-else statenents, while statements and functions. Each block has a specific scoping, which is handled by the Symbol Table as described in \ref{Block_Structure_Symbol_Tables}. The first symbol table will start with a global scope and add a scope for each new block that is being created. In table \ref{block}, we can see that we can look back to the previously known variable declarations.
\begin{table}[H]
    \begin{center}
    \begin{longtable}[c] { r c }
        $[Block]$ 
        & 
        \( \dfrac{TE' \vdash S : OK} 
        {TE \vdash D_v S : OK} \) 
        \\ \\
        & 
        \( {where \ TE' = TE(D_v, TE)} \)
    \end{longtable}
    \caption{}\label{block}
        \end{center}
\end{table}

Table \ref{active-block} is the active block which is the block we’re currently in. Unlike table \ref{block}, it cannot know its previous blocks, however, it is the one which are current for each block.
\begin{table}[H]
    \begin{center}
    \begin{longtable}[c] { r c }
        $[Block_{current}]$ 
        & 
        \( \dfrac{TE \vdash D_{v} : OK \ TE \vdash S : OK} 
        {TE \vdash current \ D_{v} \ S \ end  :  OK} \)
    \end{longtable}
    \caption{}\label{active-block}
        \end{center}
\end{table}
