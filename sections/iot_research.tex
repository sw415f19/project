\section{IoT Research}

With the rise of cheap microprocessors and humans innovation came an abundance of new small internet connected devices. These devices are what is called the "Internet of Things". Gadgets, machines and appliances which are connected to the internet\cite{iot-definition}. A subgroup within the IoT-subject is Home Automation. Home automation is the process for consumers to automate their home. Some examples are garage doors, light bulbs, thermostats, smart locks, outlets and even curtains. In a research of the smart home market, it was clear that in 2017 the worldwide customer spending on smart home was \$84 billion and projected to grow to \$155 billion by 2023\cite{home-auto-growth}. For the hometinkere and spare time hackers these products are also available. It started with the Arduino in 2003 which had hardware that was cheap and easy to program\cite{arduino-history}. It wasn't until later that the Arduino gained WiFi capabilities. Pair the internet capabilities with the fact that the easy to learn Arduino C programming could be used to programming the chips and help from the online community and new ideas form. 
In the later years the ESP8266 (2014) and ESP32 (2016) became really popular because they costs around \$1 and comes with an included WiFi module. On top of that, they can even be supported by the Arduino IDE and programmed in Arduino C with the help of some drivers by the manufacturer.

From a business standpoint, many have started making their products IoT-devices. Most of which can be controlled directly from your phone.  It really opened up for new possibilities for doing stuff without having to do them manually. Since creating all the devices, multiple companies have opened up for integrating with the home automation devices to lure users onto their platform. Google' Assistant\cite{google-assistant}, Apple's Home\cite{apple-homekit} and Amazon's Alexa\cite{amazon-alexa} currently are the major players in this. They all allow the user to control their automated home in the way the users want it automated. May it simply be lights to turn on (Eg. Ikea Trådfri bulbs), temperature regulation from night to day (ecobee thermostat) to smart connected fridges (Eg. Samsung Family Hub fridges) amoungst other things. 