
\section{Programming Paradigm}
\label{programmingparadigm}
To get an understanding of the programming style and paradigm, we must consider the user group. It is essential to create a language, which is both easy to learn and still rich in features to complete its task. Many languages today are designed for a specific problem domain – an example is Lisp which is useful for symbolic data manipulation and having a complex data structure. The programming language C is good for low-level systems programming, as the programmer has a high level of control over system memory, although an assembly-like language would be too challenging and inefficient for a novice programmer to write in.
As the paradigm is quite essential to any given programming language, it is important to consider which paradigm would best suit the needs of our problem domain.\\
\\
From the earlier PACT analysis, we have an idea of the technical competences of our users. Keeping in mind that our users probably won't have the greatest knowledge of programming, we can consider that our language to also have some educational implications. Naturally, we should therefore also look at some earlier languages that sought to accomplish the same goals of being easy to learn and understand.
One of the early programming languages which had success is BASIC, which is a high-level programming language and a part of the imperative programming paradigm. Although it did see some success, BASIC is also notorious for its spaghetti code, as the earlier versions were unstructured (both due to the simple block-structure, but also due to the lack of abstracting code away with procedures), though it was later remedied. As such, we should consider making a nice structure, so we don't end in the same situation as early BASIC programmers, who had trouble finding their way around their code, due to an overuse of goto and the lack of "nice" blocks/procedures
Another programming language created for the educational purposes in 1967, named Logo. Logo is a programming language, which is in the family with Lisp. As such, it fits into the functional programming paradigm. \cite{scott2000programming} \\
\\
To bring a typical pattern in the programming languages which got mentioned in this section, all of them got a common factor, that they run on procedural paradigm, which is the opposite of the modern object-orientated paradigm, which is being used by many of the prominent languages today (Java, C\# etc.). Why is it, that object orientated paradigm is not well used for novice users, for educational purposes? \\
\\
In an article by Michael Kölling from Monash University which research has been about why is it hard to teach object-oriented languages, while it is widely used in education and industry today. The results Kölling got from the test, was that there was a general trends in all his tests, where object oriented programming itself is a very powerful and valuable teaching tool, it had a common trend of having problems with the language and system they used. Kölling report also state, that most of their studies reported some diffuclties of the users, when they switched from producdural paradigme approach towards object oriented.\cite{fuk1} \\
\\
%\begin{itemize}
% \item Object orientated paradigm was not the %first paradigm, and at most time is not even an appropriate paradigm for smaller tasks however, it is one of the more prevalent paradigms in many industries today.
 %\item Object orientation thinking requires %much work and can be referred to as English essays, which most students do not particularly enjoy learning to write.\cite{tutlisp}\cite{medlisp}
%\end{itemize}
Reflecting through this, object orientated paradigms might not be a good first language, as it got a more profound learning curve than a procedural paradigm. Object orientation requires skills to understand, where imperative programming is straight forward without requiring no knowledge of classes and objects. It may be hard to switch thought process between imperative and object oriented programming, however, object oriented principles may be harder to learn. In another study made by people from various universities, have made a 16 weeks user study, where they found that the test participants had it difficult about learning and using object oriented techniques, where they had to classify the concepts into different categories, and provide an account for the relative importance of each concept.\cite{fuk2}\\
\\
As the language should be easy to learn, without taking a lot of time to get into the object oriented principles and classifications, an imperative language has been chosen for the Ezuino programming language.