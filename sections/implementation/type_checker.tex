\section{Type checker}
The type checker implements the semantic rules of Ezuino specified in chapter \ref{semantics}. \\
We split the type checker into two visitors TypeCheckerVisitor and ReturnStatementVisitor. TypeCheckerVisitor checks that the condition in a boolean statement has the correct type and that expressions are legal. ReturnStatementVisitor ensures that the actual returned value of a program matches the type defined in the function declaration.
The way TypeCheckerVisitor type checks expressions is through a setup where every expression node extends that inherits from an interface called ITypeNode. The ITypeNode interface requires that every child class have the method for getting and setting a type attribute. When type checking happens, we then compare each node through the checkType method.
\begin{lstlisting}[caption={checkType function}, label={code:TC:checkType}]
private void checkType(ITypeNode leftNode, ITypeNode rightNode) {
    Type leftType = leftNode.getType();
    Type rightType = rightNode.getType();
    if (leftType == null) {
        System.err.println("Left type null!");
        return;
    }
    if (rightType == null) {
        System.err.println("Right type null!");
        return;
    }
    if (leftType != rightType) {
        errorHandler.typeMismatch(leftNode, rightNode);
    }
}
\end{lstlisting}
\noindent\newline

The boolean condition is checked in an if- and else statements.
\begin{lstlisting}[caption={Visit if statement node}, label={code:TC:if}]
$$@Override
public void visit(If_stmtNode node) {
    node.getExpr().accept(this);
    node.getIfBlock().accept(this);
    if (node.getElseBlock() != null) {
        node.getElseBlock().accept(this);
    }
    checkSpecificType(node.getExpr(), Type.BOOL);
}
\end{lstlisting}
\noindent\newline

An example of type checking expressions is when an additive expression is reached, both nodes are type checked.
\begin{lstlisting}[caption={Visit additive expression node}, label={code:TC:Additive}]
$$@Override
public void visit(AdditiveExprNode node) {
    node.getLeftNode().accept(this);
    node.getRightNode().accept(this);
    checkType(node.getLeftNode(), node.getRightNode());
    node.setType(node.getLeftNode().getType());
    if (node.getType().equals(Type.BOOL) ||
        node.getType().equals(Type.STRING) && node.getOperator().equals("-")) {
        errorHandler. invalidOperatorForType(node.getOperator(), node.getType());
    }
}
\end{lstlisting}
\noindent\newline

In a unary expression, it is also necessary to check that the operations performed are done with legal types. If “\texttt{-}” is used on nodes that do not have the type integer or double or if “\texttt{!}” is used on anything than nodes with the type boolean it notifies that there is an error to the errorhandler.
\begin{lstlisting}[caption={Visit unary expression node}, label={code:TC:Unary}]
$$@Override
public void visit(UnaryExprNode node) {
    node.getNode().accept(this);
    Type nodeType = node.getNode().getType();
    node.setType(nodeType);
    String nodeOperator = node.getOperator();
    boolean printErr = false;
    if ("-".equals(nodeOperator)) {
        printErr = nodeType.equals(Type.BOOL) || nodeType.equals(Type.STRING);
    } 
    else if ("!".equals(nodeOperator)) {
        printErr = !nodeType.equals(Type.BOOL);
    }
    if (printErr) {
        errorHandler.invalidOperatorForType(nodeOperator, nodeType);
    }
}
\end{lstlisting}
\noindent\newline

When a return statement is returned it is set to its expression, this is important for type checking the return statement with the function declaration later in ReturnStatementVisitor.
It is also here that return in void functions is implemented. If a return statement has no value,we assume it to be a void function. This in practice makes the return key-word a way to preemptively end void functions.
\begin{lstlisting}[caption={Visit return statement node}, label={code:TC:return}]
$$@Override
public void visit(Return_stmtNode node) {
    Type nodeType = Type.VOID;
    AExpr expression = node.getReturnExpr();
    if (expression != null) {
        expression.accept(this);
        nodeType = expression.getType();
    }
    node.setType(nodeType);
}
\end{lstlisting}
\noindent\newline

