\section{Problem scoping}
This section will be scoping the problem and the product onwards based on the MoSCoW analysis, and the decisions described in this section.  \\
\\
The programming language will create a full code generation, which will be mainly focused and limited to Arduino primary programming language, C. The project will only be fully implemented in this language; however, some Java Bytecode generation will be made as well, but not to the full extent. As there is a lot of IoT devices, and other Arduino models, the project will be limited to be developed and tested on an Arduino Uno.  \\
Syntax of the programming language will be limited to the analysis part, which will take part in the most common syntax used in programming languages today. The functionality and syntax of the programming language will only be limited to the most crucial parts mentioned in the MoSCoW analysis. All the additional functionality, like Java’s Array List, will not be within this project scope at this time but will be available for further development. \\
As it is essential that the compiler does not contain errors, testing will get focused to a greater extent of this project. Testing of the program, JUnit test should be done in the significant fields of code and functionality, together with running automated Continuous Integration (CI), which will run the tests in every build, together with a PR/Code Quality manager running as a static code analyser. Further testing, such as code coverage test will not get considered in this development phase.  \\
\section{MoSCoW}
%-- MoSCoW til language features (Use MoSCoW to clarify the prioritization of the languages features ex. datasctructures, if else for etc.)


This section will provide an account for the different elements, which will be prioritised within a MoSCoW analysis. MoSCoW analysis is an abbreviation for Must have, Could Have, Would have and Would have but no at this time. This analysis will provide an account and ensure that the critical elements in the programming language will be included in the project. 

The prioritizing of this analysis is based on initial user interviews and pre-analysis, together with developer opinions, the MoSCoW analysis will be prioritized as the following:\\  

\subsection{Must have}
\textbf{Data Types} 

\begin{itemize} 
    \item Integer
    \item Double
    \item String
    \item Boolean
\end{itemize} 
\\
\textbf{Binary Operators} \\
\begin{itemize} 
    \item Addictive
    \item Multiplicative
    \item Division
    \item Plus
    \item Minus
    \item Less
    \item LessThanOrEqual
    \item Greater
    \item GreaterThanOrEqual
\end{itemize} 
\\
\textbf{Unary Operators} \\
\begin{itemize} 
    \item Not
    \item Minus
\end{itemize}
\\
\textbf{Control Structures} \\
\begin{itemize} 
    \item If
    \item Else
    \item While
    \item Return
\end{itemize} 
\\
\subsection{Should have }
\textbf{Data Types }
\begin{itemize} 
    \item Booleans
\end{itemize} 
\\
\textbf{Binary Operators} 
\begin{itemize} 
    \item And
    \item Or
\end{itemize} 
\\
\textbf{Homogeneous Collection }
\begin{itemize} 
    \item List
\end{itemize} 
\\
\subsection{Could have}
\textbf{Data Types }
\begin{itemize} 
    \item Char
\end{itemize} 
\\
\subsection{Would have but not at this time }
\\
\textbf{Data Types }
\begin{itemize} 
    \item Floats
    \item Unsigned Numbers
\end{itemize} 
\\
\textbf{Bitwise Operators }
\begin{itemize} 
    \item All bitwise shifts and operations. 
\end{itemize} 

 

 

 

 

 

 

 

\\ 
As the scoping reach an end, we can start putting a problem definition together, where we can define the exact problem that we in which can solve with this programming language.  