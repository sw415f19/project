\section{Contextual Restraints}
Contextual restraints in a programming language are scope rules, type binding, and block structure. This section will describe what these are and how they will be implemented in Ezuino.
\subsection{Static vs Dynamic Scope Rules}
Scope rules declare what bindings are used during the execution of a procedure.
When a function is statically scoped, it means that it saves the function and variable environment at declaration time. So, in contrast to dynamic scope rules, even if the variables or functions it uses are changed, the function itself will not be changed since it uses the environment from when it was declared. Most programming languages today use static scope rules \cite{syntax-and-semantics}. % p 85
\subsection{Explicit vs Implicit Type Declaration}
The type of a declaration can be inferred from explicit declarations at compile time or implicit declarations in run time. Explicit declaration requires the programmer to declare the type when declaring a variable, while implicit declaration binds a type to the variable when it is first assigned a value. Both explicit and implicit declaration uses static type binding \cite{conceptsOfProgrammingLanguages}.

\subsection{Static vs Dynamic Type Binding}
Static and dynamic type binding defines when type should be allowed to change. When a programming language uses static type binding it will remember the type a value was declared and not allow it to change. Dynamic type binding, on the other hand, allows the variable to change types\cite{conceptsOfProgrammingLanguages}.
Static type binding has several advantages over dynamic type binding
\begin{itemize}
    \item It has more reliable at error detection. Since variables remember the type they were first declared or assigned it can use this type for type checking. 
    \item Dynamic type binding runs slower. Execution time is slower since types must be assigned at run time. Languages with dynamic type binding must also use a run time descriptor to ensure that the variable has enough storage. The run time descriptor will keep track of the current type size of the variable, which can vary during execution since types can be changed and different types have different sizes, The execution is also slower since type checking has to be done while executing and cannot be done at compile time\cite{conceptsOfProgrammingLanguages}.
    \item Since the type of variables are only known at execution time Dynamic Type Binding is only suited for interpreters, since it is not possible to generate machine instruction without knowing the data types.
\end{itemize}
\subsection{Block Structure and Symbol Tables}\label{Block_Structure_Symbol_Tables}
In this section we will describe what symbol tables are and the different block structures they can implement.
Block structures are divided between three types, monolithic, flat and nested \cite{compilers-and-intepreters}. \\ %chapter 5
In a monolithic block structure the entire program is one block. This means that there may be no variables with the same identifier.
In a flat block structure, there is distinction between.
\begin{itemize}
    \item Local scope. Applied occurrence of locally declared identifiers are restricted to a particular block.
    \item Other declaration are global in scope. Applied occurrences of locally declared identifiers are allowed anywhere in the program. 
    In effect, the program as a whole is a block, enclosing all the other blocks.
\end{itemize}
Finally there is the nested block structure. In a nested block structure blocks can be nested inside other blocks. 
\begin{itemize}
    \item Declarations in the outermost block are global in scope. We say that the outermost block is at scope level 1.
    \item Declarations inside an inner block are local to that block. Every inner block is completely enclosed by another block. If enclosed by the outermost block, we say that the inner block is at scope level; if enclosed by a level-2 block, we say that the inner block is scope level 3; and so on.
\end{itemize}
Block structure is implemented in a programming language by the symbol table. The purpose of the symbol table is to keep track of scopes and save variables or functions declaration type and their identifier when they are declared. The type of block structure then decides how many scopes the symbol table should be designated to considerate.\\
The symbol table in a programming language has various uses. Some of the most common uses are for ensuring that variables are not declared twice in the same scope and type checking that return statements matches the type declared in the function declaration.

\subsection{Contextual Restraints For Ezuino}
The target group for Ezuino is people with no prior coding experience [\ref{targetedGroup}]. We have chosen static scope rules since it is more commonly used today and to be consistent with static type binding, which we have also chosen for Ezuino. The biggest advantage if we had chosen dynamic type binding is its increased ease of cost in terms of training new programmers that are not familiar with basic data types. While this fits our target group the disadvantage of less reliable error detection could lead to about the same amount of frustration if not worse, for someone inexperienced in writing code.\\
We have chosen explicit declaration for the same reason, if the users have to declare what type a variable is they are more aware of its type and less likely to use them incorrect. \\
In terms of block structure even though Monolithic block structure is the simplest concept to understand for new programmers, it is less flexible especially regarding the readability of control statements. Declaring variables in the block they are used, makes for more readable code. For the same reason, nested block structure is more suitable than flat block structure because it allows multiple nested control statements to have their variables declared inside their block.
The variable declarations have been limited to be at the very top of a block. In that way, the variables are always found in the same place. It takes inspiration from the programming language Pascal in that way that pascal must define variables in a special block. While the special block in Pascal is good for separation, it seems excessive to explicitly make a special block structure for declarations. As such, the Ezuino language still requires the user to specify variables in the beginning, but without the special block. \\
In Ezuino nested block structure will not include functions. Functions are limited to be declared in the global scope. This ensures that functions cannot be defined within other functions. The structure of a function is somewhat complex as a function can be very long and if Ezuino had nested block structures for functions it can be hard to distinguish which functions are available at a given time.
