\section{Semantics (and structural operational semantics figure) (EMPTY)}
The operational semantics are the rules that give our syntax meaning, and without them it would be impossible to make a working language, as a given syntax wouldn't have a clearly defined functionality. For this reason, we will define the operational semantics in this section.

There are 2 different ways of describing semantics of a language. Big-step semantics and small-step semantics. While they are equivalent, there are some advantages and disadvantages to both of them. The big-step semantics evaluates the whole expression from the start to the end expression. Small-step semantics may only evaluate part of the expression one step at a time. The disadvantage of small-step semantics is that it can be daunting to both write and read.

We have decided to make our operational semantics a small-step semantic, as a small-step semantic ruleset is easier to extend in the case of a new feature compared to a big-step semantic ruleset. This means that it would be possible to add parallelity, if we realize that we need it at a later date.


How to do the abstract expressions? make abstract after each evaluation or bruteforce our way through rules
\subsection{Arithmetic Expressions}
In the following table, it should be noted that most transitions will be on the form \( \langle envl, sto, a \rangle \Rightarrow_e \langle envl', sto', v \rangle \). This is supposed to be read as "By evaluating a in accord with the runtime stack and the store like an expression, a is evaluated to the value v, and the runtime stack and store is updated

\begin{longtable}[c] { r c }
  \centering
  
  [VAR] & \( envl, sto \vdash x \Rightarrow_e v \) \\
  & if \(env_v x = l \) \\
  & and \(sto \ l = v \) \\
  & where \(envl = (env_v, env_p) : envl' \) \\
  &  \\

  [DCL] & \( 
    \dfrac{ \langle D_V, envl[x \mapsto l][next \mapsto \text{new} \ l], sto \rangle \Rightarrow_{DV} \langle envl', sto \rangle}
    { \langle T x, envl, sto \rangle \Rightarrow_{DV} \langle envl', sto \rangle } \) \\
  & \\

  [PARENT-1] & 
    \( \dfrac { \langle e, envl, sto \rangle \Rightarrow_e \langle e', envl', sto' \rangle }
      { \langle (e), envl, sto \rangle  \Rightarrow_e \langle (e'), envl', sto' \rangle } \) \\
  & \\

  [PARENT-2] & 
    \( \langle (v), envl, sto \rangle \Rightarrow_e \langle v, envl, sto \rangle \) \\
  & \\

  [PLUS-1] & 
    \( \dfrac { \langle e_1, envl, sto \rangle \Rightarrow_e \langle e_1', envl', sto' \rangle )}
      {\langle e_1 + e_2, envl, sto \rangle \Rightarrow_e \langle e_1' + e_2, envl', sto' \rangle } \) \\
  & \\

  [PLUS-2] & 
    \( \dfrac { \langle e_2, envl, sto \rangle \Rightarrow_e \langle e_2', envl', sto' \rangle }
      {\langle v_1 + e_2, envl, sto \rangle  \Rightarrow_e \langle e_1 + e_2', envl', sto' \rangle } \) \\
  & \\

  [PLUS-3] & 
    \( \langle v_1 + v_2, envl, sto \rangle \Rightarrow \langle v, envl, sto \rangle \) \\
  & where \( v = v_1 + v_2\) \\
  & \\

  [MINUS-1] & 
    \( \dfrac { \langle e_1, envl, sto \rangle \Rightarrow_e \langle e_1', envl', sto' \rangle }
      {\langle e_1 - e_2, envl, sto \rangle \Rightarrow_e \langle e_1' - e_2, envl', sto' \rangle } \) \\
  & \\

  [MINUS-2] & 
    \( \dfrac { \langle e_2, envl, sto \rangle \Rightarrow_e \langle e_2', envl', sto' \rangle }
      {\langle v_1 - e_2, envl, sto \rangle \Rightarrow_e \langle v_1 - e_2', envl', sto' \rangle } \) \\
  & \\

  [MINUS-3] & 
    \( \langle v_1 - v_2, envl, sto \rangle \Rightarrow \langle v, envl, sto \rangle \) \\
  & where \( v = v_1 - v_2\) \\
  & \\

  [MULT-1] & 
    \( \dfrac { \langle e_1, envl, sto \rangle \Rightarrow_e \langle e_1', envl', sto' \rangle }
      {\langle e_1 * e_2, envl, sto \rangle \Rightarrow_e \langle e_1' * e_2, envl', sto' \rangle } \) \\
  & \\

  [MULT-2] & 
    \( \dfrac { \langle e_2, envl, sto \rangle \Rightarrow_e \langle e_2', envl', sto' \rangle }
      {\langle v_1 * e_2, envl, sto \rangle \Rightarrow_e \langle v_1 * e_2', envl', sto' \rangle } \) \\
  & \\

  [MULT-3] & 
    \( \langle v_1 * v_2, envl, sto \rangle \Rightarrow \langle v, envl, sto \rangle \) \\
  & where \( v = v_1 * v_2\) \\
  & \\

  [DIV-1] & 
    \( \dfrac { \langle e_1, envl, sto \rangle \Rightarrow_e \langle e_1', envl', sto' \rangle }
      {\langle e_1 / e_2, envl, sto \rangle \Rightarrow_e \langle e_1' / e_2, envl', sto' \rangle } \) \\
  & \\

  [DIV-2] & 
    \( \dfrac { \langle e_2, envl, sto \rangle \Rightarrow_e \langle e_2', envl', sto' \rangle }
      {\langle v_1 / e_2, envl, sto \rangle \Rightarrow_e \langle v_1 / e_2', envl', sto' \rangle } \) \\
  & \\

  [DIV-3] & 
    \( \langle v_1 / v_2, envl, sto \rangle \Rightarrow \langle v, envl, sto \rangle \) \\
  & where \( v = \dfrac{ v_1 }{ v_2 } \) \\
  & \\

  [NEGATION-1] & 
    \( \dfrac { \langle e_1, envl, sto \rangle \Rightarrow_e \langle e_1', envl', sto' \rangle }
      {\langle -e_1, envl, sto \rangle \Rightarrow_e \langle -e_1', envl', sto' \rangle } \) \\
  & \\

  [NEGATION-2] & 
    \( \langle -v_1, envl, sto \rangle \Rightarrow \langle v, envl, sto \rangle \) \\
  & where \( v = -v_1 \)\\
  & \\

  [NUM] & 
    \( n \Rightarrow v \) \\
  & \( \text{if } \mathcal{N} [[n]] = v \) \\

  \caption{Small-step semantics for arithmetic expressions}
\end{longtable}

\subsection{Boolean Expressions}
\begin{longtable}[c] { r c }
  \centering
  [NOT-1] & \( 
    \dfrac { \langle e, envl, sto \rangle \Rightarrow_e \langle e', envl', sto' \rangle }
      { \langle !e, envl, sto \rangle \Rightarrow_e \langle !e', envl', sto' \rangle } \)
  \\
  & \\

  [NOT-TRUE] & \( 
    \langle !v, envl, sto \rangle \Rightarrow_e \langle tt, envl, sto \rangle \)
  \\
  & if \( v = ff\) \\
  & \\

  [NOT-FALSE] & \( 
    \langle !v, envl, sto \rangle \Rightarrow_e \langle ff, envl, sto \rangle \)
  \\
  & if \( v = tt \) \\
  & \\

  [AND-1] & \( 
    \dfrac { \langle e_1, envl, sto \rangle \Rightarrow_e \langle e_1', envl', sto' \rangle }
      { \langle e_1 \ AND \ e_2, envl, sto \rangle \Rightarrow_e \langle e_1' \ AND \ e_2, envl', sto' \rangle } \)
  \\
  & \\

  [AND-2] & \( 
    \dfrac { \langle e_2, envl, sto \rangle \Rightarrow_e \langle e_2', envl', sto' \rangle }
      { \langle v_1 \ AND \ e_2, envl, sto \rangle \Rightarrow_e \langle v_1 \ AND \ e_2', envl', sto' \rangle } \)
  \\
  & \\

  [AND-3] & \( 
    \langle v_1 \ AND \ v_2, envl, sto \rangle \Rightarrow_e \langle tt, envl, sto \rangle \)
  \\
  & if \( v_1 \land v_2 \) \\
  & \\

  [AND-4] & \( 
    \langle v_1 \ AND \ v_2, envl, sto \rangle \Rightarrow_e \langle ff, envl, sto \rangle \)
  \\
  & if \( \neg(v_1 \land v_2) \) \\
  & \\

  [OR-1] & \( 
    \dfrac { \langle e_1, envl, sto \rangle \Rightarrow_e \langle e_1', envl', sto' \rangle }
      { \langle e_1 \ OR \ e_2, envl, sto \rangle \Rightarrow_e \langle e_1' \ OR \ e_2, envl', sto' \rangle } \)
  \\
  & \\

  [OR-2] & \( 
    \dfrac { \langle e_2, envl, sto \rangle \Rightarrow_e \langle e_2', envl', sto' \rangle }
      { \langle v_1 \ OR \ e_2, envl, sto \rangle \Rightarrow_e \langle v_1 \ OR \ e_2', envl', sto' \rangle } \)
  \\
  & \\

  [OR-3] & \( 
    \langle v_1 \ OR \ v_2, envl, sto \rangle \Rightarrow_e \langle tt, envl, sto \rangle \)
  \\
  & if \( v_1 \lor v_2 \) \\
  & \\

  [OR-4] & \( 
    \langle v_1 \ OR \ v_2, envl, sto \rangle \Rightarrow_e \langle ff, envl, sto \rangle \)
  \\
  & if \( \neg(v_1 \lor  v_2) \) \\
  & \\

  [EQUAL-1] & \( 
    \dfrac { \langle e_1, envl, sto \rangle \Rightarrow_e \langle e_1', envl', sto' \rangle }
      { \langle e_1 = e_2, envl, sto \rangle \Rightarrow_e \langle e_1' = e_2, envl', sto' \rangle } \)
  \\
  & \\

  [EQUAL-2] & \( 
    \dfrac { \langle e_2, envl, sto \rangle \Rightarrow_e \langle e_2', envl', sto' \rangle }
      { \langle v_1 = e_2, envl, sto \rangle \Rightarrow_e \langle v_1 = e_2', envl', sto' \rangle } \)
  \\
  & \\

  [EQUAL-3] & \( 
    \langle v_1 = v_2, envl, sto \rangle \Rightarrow_e \langle tt, envl, sto \rangle \)
  \\
  & if \( v_1 = v_2 \) \\
  & \\

  [EQUAL-4] & \( 
    \langle v_1 = v_2, envl, sto \rangle \Rightarrow_e \langle ff, envl, sto \rangle \)
  \\
  & if \( \neg(v_1 = v_2) \) \\
  & \\

  [NOT-EQUAL-1] & \( 
    \dfrac { \langle e_1, envl, sto \rangle \Rightarrow_e \langle e_1', envl', sto' \rangle }
      { \langle e_1 != e_2, envl, sto \rangle \Rightarrow_e \langle e_1' != e_2, envl', sto' \rangle } \)
  \\
  & \\

  [NOT-EQUAL-2] & \( 
    \dfrac { \langle e_2, envl, sto \rangle \Rightarrow_e \langle e_2', envl', sto' \rangle }
      { \langle v_1 != e_2, envl, sto \rangle \Rightarrow_e \langle v_1 != e_2', envl', sto' \rangle } \)
  \\
  & \\

  [NOT-EQUAL-3] & \( 
    \langle v_1 != v_2, envl, sto \rangle \Rightarrow_e \langle tt, envl, sto \rangle \)
  \\
  & if \( \neg(v_1 = v_2) \) \\
  & \\

  [NOT-EQUAL-4] & \( 
    \langle v_1 != v_2, envl, sto \rangle \Rightarrow_e \langle ff, envl, sto \rangle \)
  \\
  & if \( v_1 = v_2 \) \\
  & \\

  [GREATER-THAN-1] & \( 
    \dfrac { \langle e_1, envl, sto \rangle \Rightarrow_e \langle e_1', envl', sto' \rangle }
      { \langle e_1 > e_2, envl, sto \rangle \Rightarrow_e \langle e_1' > e_2, envl', sto' \rangle } \)
  \\
  & \\

  [GREATER-THAN-2] & \( 
    \dfrac { \langle e_2, envl, sto \rangle \Rightarrow_e \langle e_2', envl', sto' \rangle }
      { \langle v_1 > e_2, envl, sto \rangle \Rightarrow_e \langle v_1 > e_2', envl', sto' \rangle } \)
  \\
  & \\

  [GREATER-THAN-3] & \( 
    \langle v_1 > v_2, envl, sto \rangle \Rightarrow_e \langle tt, envl, sto \rangle \)
  \\
  & if \( v_1 > v_2 \) \\
  & \\

  [GREATER-THAN-4] & \( 
    \langle v_1 > v_2, envl, sto \rangle \Rightarrow_e \langle ff, envl, sto \rangle \)
  \\
  & if \( \neg(v_1 > v_2) \) \\
  & \\

  [GREATER-THAN-OR-EQUAL-1] & \( 
    \dfrac { \langle e_1, envl, sto \rangle \Rightarrow_e \langle e_1', envl', sto' \rangle }
      { \langle e_1 > = e_2, envl, sto \rangle \Rightarrow_e \langle e_1' > = e_2, envl', sto' \rangle } \)
  \\
  & \\

  [GREATER-THAN-OR-EQUAL-2] & \( 
    \dfrac { \langle e_2, envl, sto \rangle \Rightarrow_e \langle e_2', envl', sto' \rangle }
      { \langle v_1 > = e_2, envl, sto \rangle \Rightarrow_e \langle v_1 > = e_2', envl', sto' \rangle } \)
  \\
  & \\

  [GREATER-THAN-OR-EQUAL-3] & \( 
    \langle v_1 > = v_2, envl, sto \rangle \Rightarrow_e \langle tt, envl, sto \rangle \)
  \\
  & if \( v_1 \geq v_2 \) \\
  & \\

  [GREATER-THAN-OR-EQUAL-4] & \( 
    \langle v_1 > = v_2, envl, sto \rangle \Rightarrow_e \langle ff, envl, sto \rangle \)
  \\
  & if \( \neg(v_1 \geq v_2) \) \\
  & \\

  [LESS-THAN-1] & \( 
    \dfrac { \langle e_1, envl, sto \rangle \Rightarrow_e \langle e_1', envl', sto' \rangle }
      { \langle e_1 < e_2, envl, sto \rangle \Rightarrow_e \langle e_1' < e_2, envl', sto' \rangle } \)
  \\
  & \\

  [LESS-THAN-2] & \( 
    \dfrac { \langle e_2, envl, sto \rangle \Rightarrow_e \langle e_2', envl', sto' \rangle }
      { \langle v_1 < e_2, envl, sto \rangle \Rightarrow_e \langle v_1 < e_2', envl', sto' \rangle } \)
  \\
  & \\

  [LESS-THAN-3] & \( 
    \langle v_1 < v_2, envl, sto \rangle \Rightarrow_e \langle tt, envl, sto \rangle \)
  \\
  & if \( v_1 < v_2 \) \\
  & \\

  [LESS-THAN-4] & \( 
    \langle v_1 < v_2, envl, sto \rangle \Rightarrow_e \langle ff, envl, sto \rangle \)
  \\
  & if \( \neg(v_1 < v_2) \) \\
  & \\

  [LESS-THAN-OR-EQUAL-1] & \( 
    \dfrac { \langle e_1, envl, sto \rangle \Rightarrow_e \langle e_1', envl', sto' \rangle }
      { \langle e_1 < = e_2, envl, sto \rangle \Rightarrow_e \langle e_1' < = e_2, envl', sto' \rangle } \)
  \\
  & \\

  [LESS-THAN-OR-EQUAL-2] & \( 
    \dfrac { \langle e_2, envl, sto \rangle \Rightarrow_e \langle e_2', envl', sto' \rangle }
      { \langle v_1 < = e_2, envl, sto \rangle \Rightarrow_e \langle v_1 < = e_2', envl', sto' \rangle } \)
  \\
  & \\

  [LESS-THAN-OR-EQUAL-3] & \( 
    \langle v_1 < = v_2, envl, sto \rangle \Rightarrow_e \langle tt, envl, sto \rangle \)
  \\
  & if \( v_1 \leq v_2 \) \\
  & \\

  [LESS-THAN-OR-EQUAL-4] & \( 
    \langle v_1 < = v_2, envl, sto \rangle \Rightarrow_e \langle ff, envl, sto \rangle \)
  \\
  & if \( \neg(v_1 \leq v_2) \) \\
  & \\
  \caption{Small-step semantics boolean expressions}
\end{longtable}
\subsection{Boolean Expressions}
\begin{table}[H]
    \centering
    \begin{longtable}[c] { r c }
    \begin{tabular}{@{}c@{}} 
    [ASS1] &
    \\
    \\
    \end{tabular}
  \begin{tabular}{@{}c@{}}   \( <x := a, sto, env_l> \Rightarrow (sto[l -> v], envl) \)  \\ \( env_v,sto \vdash n \Rightarrow v \)
  \\ \( env_v,sto \vdash n \Rightarrow v \)
  \end{tabular}
        
 \end{longtable}
    \caption{Small-step semantics for boolean assignment expressions}\label{tab:my_label}
\end{table}
        
\begin{table}[H]
    \centering
    \begin{longtable}[c] { r c }
        
        [IF-ELSE-TRUE] & \( \dfrac{\langle \texttt{if} \ b \ \texttt{then} \ S_1 \ \texttt{else} \ S_2,\ s \rangle \Rightarrow \langle S_1,\ s\rangle}{if\ s\vdash b \rightarrow_b tt} \) \\[4ex]
        
        [IF-ELSE-FALSE] & \( \dfrac{\langle \texttt{if} \ b \ \texttt{then} \ S_1 \ \texttt{else} \ S_2,\ s \rangle \Rightarrow \langle S_2,\ s \rangle}{if\ s\vdash b \rightarrow_b ff} \) \\[4ex]
        
        [IF-TRUE] & \( \dfrac{\langle \texttt{if} \ b \ \texttt{then} \ S_1,\ s \rangle \Rightarrow \langle S_1,\ s \rangle}{if \ s\vdash b \rightarrow_b tt} \) \\\\
        
        [IF-FALSE] & \( \dfrac{\langle \texttt{if} \ b \ \texttt \ S_1,\ s \rangle \Rightarrow s}{if\ s\vdash b \rightarrow_b ff} \) \\[4ex]
        
 \end{longtable}
    \caption{Small-step semantics for boolean if statement}\label{tab:my_label}
\end{table}
        
\begin{table}[H]
    \centering
    \begin{longtable}[c] { r c }
        
        [WHILE] & \( \langle \texttt{while(} \ b\texttt{) \{} S \texttt{\}},\ s \rangle \Rightarrow \langle \texttt{if} \ b \ \texttt{then} (S\texttt{;while (} b \texttt{) \{} S\texttt{\}}) \texttt{else skip,}\ s\rangle) \) \\[4ex]
        
 \end{longtable}
    \caption{Small-step semantics for boolean while statements}\label{tab:my_label}
\end{table}
        
\begin{table}[H]
    \centering
    \begin{longtable}[c] { r c }
    
        [END] & END\\
    \end{longtable}
    \caption{Small-step semantics for end}\label{tab:my_label}
\end{table}