\subsection{Statements}

\begin{longtable}[c] { r c }
  \centering
  [ASS1] & \( \langle x := a, sto, envl \rangle \Rightarrow \langle sto[l \mapsto v], envl \rangle \) \\ 
  & where \(envl  = (env_V, env_f) : envl' \) \\
  & and \(l = env_v x \) \\
  & and \(env_V, \ sto \vdash a \rightarrow_a v \) \\
  & \\

  [IF-ELSE-1] & \( 
    \dfrac { \langle e, sto, envl \rangle \Rightarrow_e \langle e', envl', sto' \rangle }
      { \langle if (e) C_1 else C_2, sto, envl \rangle \Rightarrow_e \langle if (e') C_1 else C_2, envl', sto' \rangle } \) \\
  & \\

  [IF-ELSE-2] & \( 
    \langle if (tt) C_1 else C_2, sto, envl \rangle \Rightarrow_S \langle C_1, envl', sto' \rangle \) \\
  & \\

  [IF-ELSE-3] & \( 
    \langle if (ff) C_1 else C_2, sto, envl \rangle \Rightarrow_S \langle C_2, envl', sto' \rangle \) \\
  & \\

  [IF-1] & \( 
    \dfrac { \langle e, sto, envl \rangle \Rightarrow_e \langle e', envl', sto' \rangle }
      { \langle if (e) C, sto, envl \rangle \Rightarrow_C \langle if (e') C, envl', sto' \rangle } \) \\
  & \\

  [IF-2] & \( 
    \langle if (tt) C, sto, envl \rangle \Rightarrow_C \langle C, envl', sto' \rangle \) \\
  & \\

  [IF-3] & \( 
    \langle if (ff) C_1, sto, envl \rangle \Rightarrow_C \langle skip, envl', sto' \rangle \) \\
  & \\

  [WHILE] & \( 
    \langle while (e) C, sto, envl \rangle \Rightarrow_C \) \\
  & \(\langle if (e) \{C\ while (e')\ C\} else skip, envl', sto' \rangle \) \\
  & \\

  [SKIP] & \( 
    \langle skip, sto, envl \rangle \Rightarrow_S \langle sto, envl \rangle \) \\
  & \\

  [BLOCK-1] & \( 
    \dfrac{ \langle D_V, sto, envl \rangle \Rightarrow \langle envl', sto' \rangle }
      { \langle \{ D_V \ C \}, sto, envl \rangle \Rightarrow \langle \text{active} \ C \ \text{end}, sto', envl' \rangle} \) \\
  & where \(env_P' = \text{upd}_P(D_P, env_V', env_P) \) \\
  & where \(envl' = (env_V' env_P') : envl\) \\
  & where \(envl = (env_V, env_P) : envl'' \) \\
  & \\

  [BLOCK-2] & \( 
    \dfrac{ \langle C, sto, envl \rangle \Rightarrow \langle C', envl', sto' \rangle }
      { \langle \text{active} \ C \ \text{end}, sto, envl \rangle \Rightarrow \langle \text{active} \ C' \ \text{end}, sto', envl' \rangle} \) \\
  & \\

  [BLOCK-3] & \( 
    \dfrac{ \langle C, sto, envl \rangle \Rightarrow \langle envl', sto' \rangle}
      { \langle \text{active} \ C \ \text{end}, sto, envl \rangle \Rightarrow \langle C, sto', envl' \rangle} \) \\
  & \\

  [COMP-1] & \( 
    \dfrac{ \langle C_1, sto, envl \rangle \Rightarrow \langle C_1', envl', sto' \rangle}
      { \langle C_1;C_2, sto, envl \rangle \Rightarrow \langle C_1';C_2, sto', envl' \rangle} \) \\
  & \\

  [COMP-2] & \( 
    \dfrac{ \langle C_1, sto, envl \rangle \Rightarrow \langle envl', sto' \rangle}
      { \langle C_1;C_2, sto, envl \rangle \Rightarrow \langle C_2, sto', envl' \rangle} \) \\
  & \\

  \caption{Small-step semantics for statements}
\end{longtable}
        


As previously mentioned, the function environment is a partial function from the name of the function to the 4-tuple of a sequence of commands, a set of parameters, a variable environment, and a function environment.

\begin{table}[H]
    \centering
    \begin{longtable}[c] { r c }
        \([FUNC-DEC]\) & \( \dfrac{env_v \vdash \langle D_F, env_F[f \mapsto (S, P, env_v, env_F)] \rangle \Rightarrow_{DF} env_F' } %lower half of dfrac below
        {env_v \vdash \langle func\ T \cup \epsilon\ f(T\ P_1, T\  P_2,...,T\ P_n)\ C\ D_F, env_F \rangle \Rightarrow_{DF} env_F'} \) \\
        & where \(P = \{P_1, P_2,..., P_n\}\)\\
        & and \(n \geq 0 \) \\
        & and \(T \in \{int, boolean, double, string\}\)\\
        & \\
        \([FUNC-CALL]\) & \( \langle f(e_1,...,e_n), sto[return \mapsto  l_ret][next \mapsto new l_{ret}], envl \rangle \Rightarrow\)\\
        & \(\langle active\ C\ end, sto, (env_v'[l_1 \mapsto v_1,..., l_n \mapsto v_n, return \mapsto l_{ret}],\)\\
        & \(env_F'[f \mapsto (C, P, env_v', env_F')]) : envl \rangle \)\\
        & where \(env_F f = (C, P, env_v', env_F')\)\\
        & and \(envl = (env_v, env_F) : envl'\)\\
        & and \(envl, sto \vdash e_1,...,e_n \rightarrow v_1,...,v_n\)\\
        & and \(env_v P_1 = l_1,..., env_v P_n = l_n\)\\
        & and \(n \geq 0\)\\
        \([RETURN]\) & \( envl', sto' \vdash \langle return\ e \rangle \Rightarrow \langle envl[return \mapsto l_{ret}], sto'[l_ret \mapsto v_ret] \rangle \) \\
        & where \( envl', sto' \vdash e \rightarrow_e v_ret \)\\
    \end{longtable}
    \caption{Small-step semantics for function declarations and calls}\label{sem:func}
\end{table}
Note that the bindings of the RETURN rule are the same as in the FUNC-CALL rule, and that the special "return" name is used to store the return value from the function.