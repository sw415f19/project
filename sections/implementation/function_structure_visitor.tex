\section{Function Structure Visitor}
The purpose of this section is to explain, how the Ezuino compiler guarantees that any function called, always has the correct structure for the given function. Before a function can be called, it must have been declared first. Except for predefined functions such as Arduino functions.
\\\\
To accomplish this goal, the Ezuino compiler implements a Visitor named: “FuncStructureVisitor”. The goal for this Visitor is to verify that all functions called in a given program, always has the same structure as the function declaration. This also includes predefined Arduino functions. However, since Arduino functions are predefined by the compiler, the structure of these functions are hard coded into the Visitor.
\\\\
Exactly like the other Visitors in the compiler the FuncStructureVisitor runs through the generated AST, starting at the first node and visiting all of its children nodes.\\
The nodes that mainly will be affected by the FuncStructureVisitor, is the nodes that has to do with functions. Specifically, these nodes are function declarations, function calls and predefined functions such as Arduino functions.
\\\\
Listing \ref{FuncDefNode} shows how a function declaration node is handled by the visitor. Since the node is a function declaration, the node must be stored in a symbol table for the current scope. this is done so that when the visitor meets the function call node, it can look up in the symbol table to find how the function was declared.\\
As seen on line 3, the function declaration is stored in the symbol table with the function id and the node.
\begin{lstlisting}[caption={Visit method for FuncDefNode in FuncStructureVisitor}, label={FuncDefNode}]
    $$@Override
    public void visit(Func_defNode node) {
        symtable.enterSymbol(node.getId(), node);
        node.getBlockNode().accept(this);
    }
\end{lstlisting}
\noindent\newline
Listing \ref{customFuncCall} shows how a custom function call is handled by the visitor. The name CustomFuncCall tells us that this is not one of the predefined functions in the compiler, but instead a function that has been declared by a user.\\
Since this is a custom function call, the compiler does not know what the structure of the function is. therefore, it must search the symbol table for a function with the same id as the call.\\
Once the correct declaration has been found, the visitor compares the parameters of the function call with the parameters of the function declaration. If these does not match, an error has occurred and is thrown to the errorHandler.\\
Similarly, if the visitor cannot find a function declaration with the same id as the function call, an error has occurred and is thrown to the errorHandler.
\begin{lstlisting}[caption={Visit method for CustomFuncCallStmtNode in FuncStructureVisitor}, label={customFuncCall}]
$$@Override
public void visit(CustomFuncCallStmtNode node) {
    Func_defNode funcdef = (Func_defNode) symtable.retrieveSymbol(node.getId());
    matchParameterList(node.getId(), node.getParameters(), funcdef.getParameters());
\end{lstlisting}
\noindent\newline
Listing \ref{matchParameterList} is the private helper method, which compares parameters of function calls. It is the same method used in line 4 of listing \ref{customFuncCall}.\\
As the function structure visitor’s job is to verify that the structure of function is correct, this private method is one of the core functions of the visitor.\\
The idea behind verifying the structure of functions is to compare the types of the parameters of the function. This involves comparing the parameters of the function call and the function declaration. For the function call to be correct, the parameters must match the parameters of the function declaration.\\
To do this, the method takes a list of function call parameters and a list of function declaration parameters as input.
\\\\
First, the method compares the number of parameters of both the call and declaration. If they are not the same, an error has occurred and is thrown to the errorHandler. This happens on line 3-5 of the listing.\\
Secondly it compares the types of the parameters, in order, of both the call and declaration. If they are not the same, an error has occurred and is thrown to the errorHandler. This happens on line 7-11 of the listing.
\begin{lstlisting}[caption={Private helper method for verifying function parameters in FuncStructureVisitor}, label={matchParameterList}]
    private void matchParameterList(String functionname, List<AExpr> callparams, List<DclNode> defparams) {
        int parametercount = callparams.size();
        if (parametercount != defparams.size()) {
            errorHandler.parameterLengthError(functionname);
            return;
        }
        for (int i = 0; i < parametercount; i++) {
            AExpr callParam = callparams.get(i);
            DclNode dclnode = defparams.get(i);
            if (callParam.getType() != dclnode.getType()) {
                errorHandler. parameterTypeError(functionname);
            }
        }
    }
\end{lstlisting}
\noindent\newline
Listing \ref{AnalogReadNode} is an example of a predefined function in the compiler and shows how the visitor generally handles predefined function calls.\\
In this case the function call is to AnalogRead, which is an Arduino specified function. In the case of AnalogRead, it is necessary to perform an additional check for the size of the int parameter, since Arduino only handles small bit sizes. The additional check occurs on line 8. This check is not present for all predefined functions.
\\\\
Since AnalogRead is a predefined function, verifying the function structure is relatively simple. The reason for this is that because the function is predefined, the compiler already knows what parameters are required for the call. Predefined functions therefore have hard coded parameters, which cannot be changed at runtime.\\
As seen in line 3, the expected type of the AnalogRead function is an int. And since there is only one type, the function only takes one parameter.\\
Additionally, the parameter must be held within either an IntegerLiteral node or a IdNode. This is to ensure that only the correct int node is the parameter for the function call. A check for this occurs on line 5-6.\\
Lastly, a call to compare the parameters is made on line 9.
\\\\
Other predefined functions are handled similarly to the AnalogRead function. The main detail that changes is how many parameters the function has and which types the parameters have.\\
Additionally, only the Arduino functions that use small bit sized ints are required to have the int size check.
\begin{lstlisting}[caption={Visit method for AnalogReadNode in FuncStructureVisitor}, label={AnalogReadNode}]
$$@Override
public void visit(AnalogReadNode node) {
    Type[] expectedType = {Type.INT};
    AExpr firstParam = node.getParameters().get(0);
    if (!(firstParam instanceof IntegerLiteral || firstParam instanceof IdNode)) {
        errorHandler. invalidFunctionParameterError(node.getID());
    }
    checkArduinoRange(firstParam);
    checkFuncParameters(node.getID(), node.getParameters(), 1, expectedType);
}
\end{lstlisting}
\noindent\newline
Listing \ref{checkFuncParameters} shows the private helper method for comparing predefined function parameters. Similarly to the private method in listing \ref{matchParameterList}, the purpose of this function is to verify function calls and are therefore also a core part of the visitor.\\
The idea behind the checkFuncParameters method is the same as the matchParameterList. However, since this is for predefined functions, it compares function parameters with required types. For the function call to be correct, the parameters must match the predefined types.
\\\\
First, the method confirms that the amount of parameters is the same as required. If they are not the same, the function call is incorrect and an error is thrown to the errorHandler. This check occurs on line 2-3.\\
Secondly the method verifies that the parameters have the correct types in the correct order. If the types are not the same, an error is thrown to the errorHandler.\\
The comparison is done by calling another helper method, which compares two types. This check occurs on line 5-7.
\begin{lstlisting}[caption={Private helper method for verifying predefined function parameters in FuncStructureVisitor}, label={checkFuncParameters}]
private void checkFuncParameters(String nodeId, List<AExpr> parameters, int requiredParameters, Type[] typeList) {
    if (parameters.size() != requiredParameters) {
        errorHandler.parameterLengthError(nodeId);
    }
    for (int i = 0; i < requiredParameters; i++) {
        if (parameters.get(i) != null) {
            checkSpecificType(parameters.get(i), typeList[i]);
        }
    }
}
\end{lstlisting}
\noindent\newline
The five listings of code looked at in this section, should supply a general idea of the purpose of the Function Structure Visitor and how it works.\\
Not all of the visit functions were reviewed, but the few selected ones prove good examples of their implementation.
\\\\
Summing up, the simple idea of the visitor is to check for declared functions and store them. Then when a function call occurs, check which type of function has been called.\\
For custom functions calls, the visitor searches for a function declaration with the same Id as the call. If such a declaration exists, it compares the parameters of the call with the declaration.\\
For predefined function calls, the compiler already knows which type of parameters are required and can therefore perform its parameter check, without searching for a function declaration.