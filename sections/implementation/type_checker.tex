\section{Type checker visitor}
In this section Eziuno's type checking and the type checker visitor is discussed.\\
As mentioned in the start of the chapter, the purpose of the type checker is to verify types. Specifically, this means verifying that the type of AST nodes has the correct types.\\
The rules of the type checker is the semantic rules of Ezuino, which was described in chapter \ref{semantics}. The semantic rules describes which types nodes should have, in order to make an expression legal.
\\\\
The type checker implements two private functions, which it uses to compare types. As both functions are used to compare types, they are very similar to each other. The difference between them is that one compares the attributes of nodes, while the other a expected type.
\\\\
Listing \ref{code:TC:checkType} shows the private function \texttt{checkType}, used for comparing two nodes. This function is the main function used for comparing expressions. The reason behind this, is that every expression node extends from an interface named \texttt{ITypeNode}.\\
This interface allows a AST node to hold a type. The type checker then uses the \texttt{ITypeNode} of two nodes and compares them, using the rules from chapter \ref{semantics}. Listing \ref{code:TC:Additive} on line 5, shows an example of this comparison.\\
\begin{lstlisting}[caption={private function checkType in the type checker}, label={code:TC:checkType}]
private void checkType(ITypeNode leftNode, ITypeNode rightNode) {
    Type leftType = leftNode.getType();
    Type rightType = rightNode.getType();
    if (leftType == null) {
        System.err.println("Left type null!");
        return;
    }
    if (rightType == null) {
        System.err.println("Right type null!");
        return;
    }
    if (leftType != rightType) {
        errorHandler.typeMismatch(leftNode, rightNode);
    }
}
\end{lstlisting}
\noindent\newline
\input{figures/implementation/typeCecker/additiveTypeCheck}
\noindent\newline

Listing \ref{code:TC:checkSpecType} shows the private function for comparing a node with a specific type. Hence the name \texttt{checkSpecificType}. Similarly to the \texttt{checkType} function, the \texttt{checkSpecificType} function compares a node using the \texttt{ITypeNode} interface. However, instead of comparing two nodes, it compares the node with an \texttt{expectedType}.\\
This allows the type checker to compare a node, with a type directly. One place where this is useful is in logical statements since the condition always should be boolean. This is shown in listing \ref{code:TC:while}.\\
\begin{lstlisting}[caption={private function checkSpecificType in the type checker}, label={code:TC:checkSpecType}]
private void checkSpecificType(ITypeNode node, Type expectedType) {
    Type nodeType = node.getType();
    if (nodeType == null) {
        System.err.println("Compiler error. nodeType was null in Typechecker.");
    }
    if (nodeType != expectedType) {
        errorHandler.unexpectedType(node, expectedType);
    }
}
\end{lstlisting}
\noindent\newline
\begin{lstlisting}[caption={Visit for a while statement node in the type checker.}, label={code:TC:while}]
$$@Override
public void visit(While_stmtNode node) {
    node.getExprNode().accept(this);
    node.getBlockNode().accept(this);
    checkSpecificType(node.getExprNode(), Type.BOOL);
}
\end{lstlisting}
\noindent\newline

For unary expression, it is also necessary to check that the operations performed are done with legal types. If “\texttt{-}” is used on nodes that do not have the type integer or double, and if “\texttt{!}” is used on nodes with the type boolean, an error has occurred and is thrown to the errorHandler. This is shown in listing \ref{code:TC:Unary}.\\
\input{figures/implementation/typeCecker/unaryExprCheck.tex}
\noindent\newline