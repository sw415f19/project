\section{Semantics (and structural operational semantics figure) (EMPTY)}
The operational semantics are the rules that give our syntax meaning, and without them it would be impossible to make a working language, as a given syntax wouldn't have a clearly defined functionality. For this reason, we will define the operational semantics in this section.

There are 2 different ways of describing semantics of a language. Big-step semantics and small-step semantics. While they are equivalent, there are some advantages and disadvantages to both of them. The big-step semantics evaluates the whole expression from the start to the end expression. Small-step semantics may only evaluate part of the expression one step at a time. The disadvantage of small-step semantics is that it can be daunting to both write and read.

We have decided to make our operational semantics a small-step semantic, as a small-step semantic ruleset is easier to extend in the case of a new feature compared to a big-step semantic ruleset. This means that it would be possible to add parallelity, if we realize that we need it at a later date.

\begin{table}[H]
    \centering
    \begin{longtable}[c] { r c }
        \hline
        [PLUS-1] & \( \dfrac{a_1 \Rightarrow a^{'}_{1}}{a_1 + a_2 \Rightarrow a^{'}_{1} + a_2} \) \\
        
        [PLUS-2] & \( \dfrac{a_2 \Rightarrow a^{'}_{2}}{v_1 + a_2 \Rightarrow v_1 + a^{'}_{2}} \) \\
        
        [PLUS-3] & \(v_1 + v_2 \Rightarrow z \text{ where } z = v_1 + v_2\) \\
        
        [MINUS-1] & \( \dfrac{a_1 \Rightarrow a^{'}_{1}}{a_1 - a_2 \Rightarrow a^{'}_{1} - a_2} \) \\
        
        [MINUS-2] & \( \dfrac{a_2 \Rightarrow a^{'}_{2}}{v_1 - a_2 \Rightarrow v_1 - a^{'}_{2}} \) \\
        
        [MINUS-3] & \(v_1 - v_2 \Rightarrow z \text{ where } z = v_1 - v_2\) \\
        
        [MULT-1] & \( \dfrac{a_1 \Rightarrow a^{'}_{1}}{a_1 * a_2 \Rightarrow a^{'}_{1} * a_2} \) \\
        
        [MULT-2] & \( \dfrac{a_2 \Rightarrow a^{'}_{2}}{v_1 * a_2 \Rightarrow v_1 * a^{'}_{2}} \) \\
        
        [MULT-3] & \(v_1 * v_2 \Rightarrow z \text{ where } z = v_1 \cdot v_2\) \\
        
        [DIV-1] & \( \dfrac{a_1 \Rightarrow a^{'}_{1}}{a_1 / a_2 \Rightarrow a^{'}_{1} / a_2} \) \\
        
        [DIV-2] & \( \dfrac{a_2 \Rightarrow a^{'}_{2}}{v_1 / a_2 \Rightarrow v_1 / a^{'}_{2}} \) \\
        
        [DIV-3] & \(v_1 / v_2 \Rightarrow z \text{ where } z = \lfloor\frac{v_1} {v_2}\rfloor\) \\
        
        [PARENT-1] & \( \dfrac{a_1 \Rightarrow a^{'}_{1}}{(a_1) \Rightarrow (a^{'}_{1})} \) \\
        
        [PARENT-2] & \( (z) \Rightarrow z \) \\
        
        [NEG-1] & \( \dfrac{a_1 \Rightarrow a^{'}_{1}}{-a_1 \Rightarrow -a^{'}_{1}} \) \\
        
        [NEG-2] & \( -v_1 \Rightarrow z \text{ where } z = -v_1 \)\\
        
        [NUM] & \( n \Rightarrow z \text{ if } \mathbb{Z}[[n]] = z \) \\
        
        \hline
    \end{longtable}
    \caption{Caption}\label{tab:my_label}
\end{table}
\subsection{Statements}
\begin{table}[H]
    \centering
    \begin{longtable}[c] { r c }
    \begin{tabular}{@{}c@{}} 
    [ASS1] \\
    \newline \\
    \newline \\
    \newline \\
    \end{tabular}
  \begin{tabular}{@{}c@{}}   \( \langle x := a, sto, envl \rangle \Rightarrow (sto[l \mapsto v], envl) \)  \\ \( where \ envl  = (env_v, env_p) : envl' \)
  \\ \( and \ l = env_v x \)
  \\ \( and \ env_v, \ sto \vdash a \rightarrow_a v \)
  \end{tabular}
        
 \end{longtable}
    \caption{Small-step semantics for boolean assignment expressions}\label{sem:bool-ass}
\end{table}
        

\begin{longtable}[c] { r c }
  \centering
  [IF-ELSE-1] & \( 
    \dfrac { \langle e, envl, sto \rangle \Rightarrow_e \langle e', envl', sto' \rangle }
      { \langle if (e) \{S_1\} else \{S_2\}, envl, sto \rangle \Rightarrow_S \langle if (e') \{S_1\} else \{S_2\}, envl', sto' \rangle } \)
  \\
  & \\

  [IF-ELSE-2] & \( 
    \langle if (tt) \{S_1\} else \{S_2\}, envl, sto \rangle \Rightarrow_S \langle S_1, envl', sto' \rangle \)
  \\
  & \\

  [IF-ELSE-3] & \( 
    \langle if (ff) \{S_1\} else \{S_2\}, envl, sto \rangle \Rightarrow_S \langle S_2, envl', sto' \rangle \)
  \\
  & \\

  [IF-1] & \( 
    \dfrac { \langle e, envl, sto \rangle \Rightarrow_e \langle e', envl', sto' \rangle }
      { \langle if (e) \{S_1\}, envl, sto \rangle \Rightarrow_S \langle if (e') \{S_1\}, envl', sto' \rangle } \)
  \\
  & \\

  [IF-2] & \( 
    \langle if (tt) \{S_1\}, envl, sto \rangle \Rightarrow_S \langle S_1, envl', sto' \rangle \)
  \\
  & \\

  [IF-3] & \( 
    \langle if (ff) \{S_1\}, envl, sto \rangle \Rightarrow_S \langle skip, envl', sto' \rangle \)
  \\
  & \\

  [WHILE] & \( 
    \langle while (e) \{S\}, envl, sto \rangle \Rightarrow_S \) 
  \\
  & \(\langle if (e) \{S\ while (e') \{S\}\} else \{skip\}, envl', sto' \rangle \) 
  \\
  & \\

  [SKIP] & \( 
    \langle skip, envl, sto \rangle \Rightarrow_S \langle envl, sto \rangle \)
  \\
  & \\

  \caption{IF-ELSE}
\end{longtable}
        
        
\begin{table}[H]
    \centering
    \begin{longtable}[c] { r c }
    
    \begin{tabular}{@{}c@{}} 
    [IF-ELSE-TRUE] \\
    \newline
    \end{tabular}
  \begin{tabular}{@{}c@{}}   \(
  \langle if \ {b} \ then \ {S_1} \ else \ {S_2} \ sto, envl\rangle \Rightarrow \langle{S_1},sto,envl\rangle
  \)  \\ \(
  if \ env_v, \ sto \vdash b \rightarrow_b {tt} \ where \ envl \ = \ (env_v, env_p) \ : \ envl' 
  \) 
  \end{tabular}
        
 \end{longtable}
    \caption{Small-step semantics for boolean if-else-true expressions}\label{sem:if-else-true}
\end{table}

\begin{table}[H]
    \centering
    \begin{longtable}[c] { r c }
    
    \begin{tabular}{@{}c@{}} 
    [IF-ELSE-FALSE] \\
    \newline
    \end{tabular}
  \begin{tabular}{@{}c@{}}   \(
  \langle if \ {b} \ then \ {S_1} \ else \ {S_2} \ sto, envl\rangle \Rightarrow \langle{S_2},sto,envl\rangle
  \)  \\ \(
  if \ env_v, \ sto \vdash b \rightarrow_b {ff} \ where \ envl \ = \ (env_v, env_p) \ : \ envl' 
  \) 
  \end{tabular}
        
 \end{longtable}
    \caption{Small-step semantics for boolean if-else-false expressions}\label{sem:if-else-false}
\end{table}

\begin{table}[H]
    \centering
    \begin{longtable}[c] { r c }
    
    \begin{tabular}{@{}c@{}} 
    [IF-TRUE] \\
    \newline
    \end{tabular}
  \begin{tabular}{@{}c@{}}   \(
  \langle if \ {b} \ then \ {S_1} \ sto, envl\rangle \Rightarrow \langle{S_1},sto,envl\rangle
  \)  \\ \(
  if \ env_v, \ sto \vdash b \rightarrow_b {tt} \ where \ envl \ = \ (env_v, env_p) \ : \ envl' 
  \) 
  \end{tabular}
        
 \end{longtable}
    \caption{Small-step semantics for boolean if-true expressions}\label{sem:if-true}
\end{table}

\begin{table}[H]
    \centering
    \begin{longtable}[c] { r c }
    
    \begin{tabular}{@{}c@{}} 
    [IF-FALSE] \\
    \newline
    \end{tabular}
  \begin{tabular}{@{}c@{}}   \(
  \langle if \ {b} \ then \ {S_1} \ sto, envl\rangle \Rightarrow \langle{S_2},sto,envl\rangle
  \)  \\ \(
  if \ env_v, \ sto \vdash b \rightarrow_b {ff} \ where \ envl \ = \ (env_v, env_p) \ : \ envl' 
  \) 
  \end{tabular}
        
 \end{longtable}
    \caption{Small-step semantics for boolean if-false expressions}\label{sem:if-false}
\end{table}

\begin{table}[H]
    \centering
    \begin{longtable}[c] { r c }
    
    \begin{tabular}{@{}c@{}} 
    [WHILE] \\
    \newline
    \end{tabular} 
  \begin{tabular}{@{}c@{}}   \(
  \langle while \ b \ do \ S, \ sto, \ envl \rangle \Rightarrow
  \)  \\ \(
  \langle if \ b \ then \ (S; \ while \ b \ do \ S) \ else \ skip, \ sto, \ envl \rangle
  \) 
  \end{tabular}
        
 \end{longtable}
    \caption{Small-step semantics for boolean while expressions}\label{sem:while}
\end{table}
As previously mentioned, the function environment is a partial function from the name of the function to the 4-tuple of a sequence of commands, a set of parameters, a variable environment, and a function environment.
\begin{table}[H]
    \centering
    \begin{longtable}[c] { r c }
        [FUNC-DEC-1] & \( \dfrac{env_v \vdash \langle D_f, env_f[f \mapsto (S'', x, env_v, env_f)] \rangle \Rightarrow_{DF} env_f[f \mapsto S', x, env_v, env_f)] } %lower half of dfrac below
        {env_v \vdash \langle func T_{desc} f()} \)
    \end{longtable}
    \caption{Small-step semantics for function declaration}\label{sem:func}
\end{table}