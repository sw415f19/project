\chapter{Compiler Construction}
This chapter will briefly explain the different steps in compiler construction. 
 \begin{figure}[H]
\centering
\includegraphics[width=\textwidth]{figures/compiler-process.png}
\caption{Organization of a compiler, Source: Fischer \cite{crafting-a-compiler} }
\label{syntax-overview}
\end{figure}
%This chapter will contain the decisions we made in regards to the compiler of our language, as well as our arguments for making these decisions.

\section{The Scanner}
A scanner, also called an Lexer, is the first processing step of a programming language. It reads the input text, ignores comments, and turns it into a compact and uniform format known as tokens. 
A scanner is most often created through declarative programming, meaning to create a scanner you tell a generator what to scan and let the generator do the implementation. What it must scan is specified through regular expressions.

\section{Parsing}
The parser reads tokens generated by the scanner and generates phrases according to a context-free grammar file which it then checks are syntactically correct. If there is an error it either reports it, attempts to repair it (to create a syntactically valid program), or tries to recover from the error in order to continue parsing. A parser is typically created through declarative programming. The results of parsing should also be, besides verifying the syntax of a program is valid, to create an abstract syntax tree.

%Parsing is a process where it analyses a string of symbols in a computer language and then trying to understand the meaning of the data that has been constructed. Which can then be made into a parse tree because the parsing understands the relationship by how the string of symbols must be understood. 
%This section will explain what parsing is, how it works and what kind of different parsing methods exist. 


\subsection*{Top-down and bottom-up parsing}
When constructing a parser, one of two strategies is usually used to recognize and construct tokens based on the context-free grammar of the language, namely top-down parsing and bottom-up parsing.\\

Given a starting symbol and a set of rules, a top-down parser will start reading its input as a long line of tokens from a lexer. In general, the tokens are read from left to right\footnote{Unless your source language is a language that writes from right to left, of course}, and are afterwards derived from the starting symbol using a leftmost-derivation of the non-terminals through the grammatical rules\cite{conceptsOfProgrammingLanguages}.\\

An example of a top-down parsing is the recursive descent parser, which is built from a collection of subprograms that each produces its own top-down parse tree generator. The subprograms in itself are recursive and the recursive descent parser will only create one subprogram for every non-terminal it can find in the user-generated grammar\cite{conceptsOfProgrammingLanguages}.\\

A bottom-up parser also reads a line of tokens, but rather than starting at the top and deriving to terminals, it builds non-terminals based on the symbols it reads, by performing a rightmost-derivation in accord with the grammatical rules, until it reaches the starting symbol\cite{conceptsOfProgrammingLanguages}.\\

The end goal of either strategy is to generate a parse tree\footnote{Sometimes known as a concrete syntax tree (a CST)}, which represents the syntactic structure of the program source code. Like the tree data-structure, it features a root node and several tree nodes, however, due to the parsing process, it is also important to maintain ordering, so that the program is aware of the constructions the parser has invoked. As such, the CST also features children and parents. The parse tree is defined by an input string and a grammar, or in this case for the project, a context-free-grammar \\
\begin{figure}[H]
\centering
\frame{\includegraphics[width=0.7\textwidth]{figures/parse_tree_example.png}}
\caption{Parse tree example}
\label{exampleparse}
\end{figure}

The picture above shows a parse tree, where the highest node, which is A will have 2 child nodes which are B and C. It will then traverse further down until it reaches an end.

\subsection*{LL-parsing}
The LL parser (Left to right and left most derivation) is a parser that parses its input from left to right while using a top-down parsing technique. It is usually denoted as LL(k), where k is a greater than or equal to 0, where the k signifies the amount of lookahead tokens the parser can look at\cite{conceptsOfProgrammingLanguages}.


\subsection*{LR-parsing}
The LR parser (Left to right and rightmost derivation) is a parser that parses its input from left to right, but unlike the LL(k) parser, the LR parser uses a bottom-up parsing technique and produces a rightmost derivation. Just as the LL(k) the k in the LR(k)-definition signifies the amount of tokens the LR will look ahead for\cite{conceptsOfProgrammingLanguages}. LR-parsers usually have a magnitude of either 0 or 1 (LR(0) or LR(1)), as LR(1)-parsers already use many resources to parse a given input, and even higher magnitudes severely slow down the entire program.


\subsection*{LALR-parsing}
The LALR(k) parser is a parser that utilizes the same parsing strategy as the LR(k) parser (bottom-up), but with a different way to generate a parse table, the part of an LR parser that instructs the parser how to react to a given input. Usually, an LALR-parser is of magnitude 1, as an LALR(1)-parser is more powerful than an LR(0)-parser, but not as resource intensive as an LR(1)-parser. It should be noted, that the LR(1)-parser is still more powerful than the LALR(1)-parser, which is why it is still in use\cite{crafting-a-compiler}.


\subsection*{Table driven LL(K) parsing}
The table driven LL(k) parser uses a similarity analysis to the generated grammar so that when it starts with pushing the start symbol into the stack, the parser would try and find a match of symbols from the input and when found put the symbol into the top of the stack\cite{crafting-a-compiler}. This is good for when you have a code that is very big and wants the parsing to be automatically by using a stack.
\\

Now that the parsing and the different parsing methods that exist, while also understanding what a parsing tree is, the next chapter will explain what and how the abstract syntax tree works.

\subsection{Parser generator tools}
In this section, we used an explanation of what kind of parser generators exist and for the project. To understand the general concept, a parser is an interpreter/compiler, which takes sequences of tokens and translates the tokens that the scanner interprets. An example of a parser is to build a data structure for an AST to a new program language.
The first parser generator that was reviewed was ANTLR 3. ANTLR 3 can either create a tree parser, lexer or parsers. ANTLR3 also use a parsing technique named LL(*). ANTLR parser generator also has a newer, however, less documented version, ANTLR 4. This version differs from ANTLR 3 because ANTLR 4 creates a general parser tree with either listeners or visitors, which makes it possible to go through the generated tree to find its children \cite{ANTLR4-Why}. It also uses the same parsing technique as ANTLR3. The other parse tool generator that was tested, which were named Coco/R. Coco / R generates a scanner and a parser from the source language and grammar the user has programmed in. The parser that is created is an LL(k), which makes it able to peek at the next symbol k without reading it \cite{COCO/R}. \\
\\
Another parse generator tool that was reviewed, Gold. Gold generates a parser with the LALR parsing algorithm good to discover error recovery and display the information on what happened during the parse phase because the generator knows which token should be expected. GOLD parser generator also supports many programming languages which is another good aspect of it.\\
To make a parse tree with GOLD, the first step would be to write a file with the grammar for the programming language. After that, the GOLD builder will create a parse table with the LALR parsing technique and check if the created grammar contains any problems or ambiguities while saving the information into a separate file. If there are some, it will be reported. Once that is done it GOLD could read the created grammar with a parsing engine that can work in different programming languages that GOLD supports \cite{GOLD}. 
\\

In all cases, the source text is analyzed by the parser engine and a parse tree is constructed.
SableCC is another parser generator that can create abstract syntax trees and tree-walker classes. The problem with this parser generator is that the only language that the users can write it with is the programming language Java. Other programming languages are not supported with SableCC. The parsing technique SableCC uses are the LALR(1) parsing\cite{SableCC}.\\
\\
The parser generator that was used for the project was the ANTLR 4, which uses the parsing technique top-down with an LL(*) parsing. because there was a need of creating a generated AST tree that was able to walk through to check every child it has so that it could be checked of its value and because it was more user-friendly than the other parser generator tools. 

..

\section{Abstract Syntax Tree}
An abstract syntax tree provides the advantage that 
Removes non essential elements of the syntax. E.g. the colons, equality signs etc. 
Provides a structure which can be altered and annotated to make sure that contextual analysis is performed correctly.
The abstract syntax tree is more an idea since it is very dependent on the construction of the programming language. The idea whole idea is that the entire source program should be stored in the structure and using this structure alone, the compiler programmer should be able to reconstruct the entire source program (Source Bent slides).

\section{Type Checker}
During the Type Checker, the types of variables and functions are checked. By saving what the type of each variable is, it is possible to check whether the variables are correctly used. This might be to check if an assignment of the variable is possible or not. This phase is not limited to only type checking. The phase is meant for all the semantic analysis, which is dependant on the compiler programmers. This might include checking for double declaration, declaration before usage, etc.
The result of running the Type Checker is a decorated Abstract Syntax Tree. The abstract syntax tree might is decorated with types but might have other information associated with it, depending on the compiler.

\section{Translator}
The translator phase of the compiler takes the decorated abstract syntax tree and makes it into an intermediate representation. The reason for this is that it might be easier to deal with the intermediate representation in the code generation phase.
The translator phase is optional as code can be generated directly from the abstract syntax tree.
The output result is simply the intermediate representation.

\section{Optimizer}
The optimizer phase is optional. This is where the code is optimized and bottlenecks are removed. It is done with the idea in mind that you can reduce the size of the code and also reduce the running time of the program. 
\begin{itemize}
    \item Peekhole optimization
    \item Constant folding
    \item Common sub-expression elimination
    \item Code motion
    \item Dead code elimination
\end{itemize}{}

In order to make more efficient code one might eliminate dead variables like this:
\begin{verbatim}
func int magicNumber() {
    int t
    t := 40 * 20 - 100 / 4
    return 42
}
\end{verbatim}
where t is a dead variable.

\section{Code Generator}
In the Code Generator phase, the target code is generated. Depending if the compiler generated the intermediate representation or not, the source here is either the intermediate representation or the abstract syntax tree directly. The implementation here is very dependant on the target language as some target languages uses registers directly or stack based machines.

ANTLR4 only provides a context syntax tree (also known as a parse tree) of the source program. The parse tree from ANTLR is the direct translation of the source program parsed by ANTLR using the grammar. The nodes in the parse tree is thus directly related to the tokens provided in the grammar.

To convert the parse tree to a abstract syntax tree, you have to convert every single node to the alternative.
Doing this with ANTLR requires a great overview of both datastructures and the visitor which can be generated by ANTLR. The visitor is an ANTLR specific implementation of the visitor pattern.

\section{Visitor pattern}
The visitor pattern is an object oriented pattern provided by the GoF Design Patterns book.
It solves the problem of defining a new operation without changing the classes of the elements on which it operates.
The visitor therefore makes it easy to handler functions for handling the classes in specific contexts. A visitor simply has to specify how to visit all the nodes in the tree.

ANTLR's visitor has generics and implements methods for navigating the entire parse tree both up and down in the tree. That is the way ANTLR has chosen to generate the tree structure.
ANTLR's default BaseVisitor automatically provides an implementation for navigating the parse tree. Thus you only have to override the functionality for the functions where you want to create a node.
As each visit functionality has 2 responsibilities. Generate the node for the abstract syntax tree (with properties) and set the relation between nodes in the node and to it's potential children.
The BaseVisitor also has supports for generics for explicitly tell what should be returned. By simply setting this to a superclass for all nodes in the abstract syntax tree, all functions in the tree should return a node. 

In the end, the object that is returned from the ANTLR BaseVisitor is the top node of the abstract syntax tree.


\section{Symboltable}
While the visitor pattern provides a way of handling the nodes in the abstract syntax tree there might be necessary to keep some information when transitioning trough the nodes. This can simply be kept in a private attribute for simple information. However there might be some difficulty dealing when scope rules. Thus it might be nessecary with a symboltable.
A symboltable is datastructure that XX TODO. Whenever opening a new scope the symboltable needs to know and thus declares a new scope. Whenever exiting a scope, close the scope. 
You end up with a decorated abstract syntax tree.

%\subsection{Codegeneration}
%earing the end, of the compiler, after doing all the contextual analysis, you end up with a decorated syntax tree which has made sure that everything lives up to the programming language that has been defined. Now it’s time for code generation.
%Here there can be multiple code generators as it simply can follow the visitor pattern in order to generate the code. This is the point where the designer of the compiler decides what the resulting language of the compiler should be. It could be C, Assembly, etc. It depends on what the programming language is designed for. Sometimes it might be necessary to make some predefined code in order to support the structures and functionality of the programming language.
%A good approach might be to create an intermediate language if you want to compile multiple high-level programming languages to a low-level programming language. In that way instead of making many compilers from D => B, E => B, D => C, E => C, you can simply create one compiler from the high-level language to the intermediate language. Similarly, if you want to output to multiple low level languages, you simply need to add another compiler from the intermediate language to the new low-level language. In that way, instead of making high-level \* low-level number of compilers, you simply have to make Highlevel \+ Low-level number of compilers. It adds abstraction but might also add complexity if either the source language or the destination language has structures which are hard or impossible to implement. Some examples of compilers with intermediate languages are Java and C\#. Java compiles to Java Bytecode which is the instruction set of the Java Virtual Machine. C\# compiles to Common Intermediate Language which is then run on the Common Language Runtime.

\section{Compiler}
A compiler is a translator from a source language to a target language.
Depending on the compiler, the compiler may generate one or more of these: Pure Machine Code, Augmented Machine Code or Virtual Machine Code.[Ref Fischer]
In this report the source language is the language "Ezuino" and the target language is Jasmin [TODO].

\subsection{Jasmin}
Jasmin is an assembler for the Java Virtual Machine. It provides takes ASCII syntax of the Java Bytecode and makes into the binary Java Bytecode. [Fischer cap 10.2.2]. 
