\subsection{Blocks}
Blocks, can be refereed to as the body of if-else, while statements and functions. Each block has a specific scoping, which are being handled by the Symbol Table. The first symbol table will start with a global scope, and add a scope for each new block which are being created. In table \ref{block}, we can see that we look back to the previous known variable declarations. 
\begin{table}[H]
    \begin{center}
    \begin{longtable}[c] { r c }
        [Block] 
        & 
        \( \dfrac{TE' \vdash S : OK} 
        {TE \vdash D_v S : OK} \) 
        \\ \\
        & 
        \( {where \ TE' = TE(D_v, TE)} \)
    \end{longtable}
    \caption{}\label{block}
        \end{center}
\end{table}

Table \ref{active-block}, is the active block, which is the block we're currently in. Unlike table \ref{block}, it cannot know it's previous blocks, however, it's the one which are current for each block.
\begin{table}[H]
    \begin{center}
    \begin{longtable}[c] { r c }
        [Block_{current}] 
        & 
        \( \dfrac{TE \vdash D_{v} : OK \ TE \vdash S : OK} 
        {TE \vdash current \ D_{v} \ S \ end  :  OK} \)
    \end{longtable}
    \caption{}\label{active-block}
        \end{center}
\end{table}

\subsection{Literals}
A literal is where the types, which has been assigned each literal, will evaluate towards being a number or a string. They are the last part of the derivation tree.

\begin{table}[H]
    \centering
    \begin{longtable}[c] { r c }
        [Bool_{true}] & 
        \( {TE \vdash true : bool} \) \\
    \end{longtable}
    \caption{}\label{s-empty}
\end{table}

\begin{table}[H]
    \centering
    \begin{longtable}[c] { r c }
        [Bool_{false}] & 
        \( {TE \vdash false : bool} \) \\
    \end{longtable}
    \caption{}\label{s-empty}
\end{table}

\begin{table}[H]
    \centering
    \begin{longtable}[c] { r c }
        [String] & 
        \( {TE \vdash s : string} \) \\
    \end{longtable}
    \caption{}\label{s-empty}
\end{table}

\begin{table}[H]
    \centering
    \begin{longtable}[c] { r c }
        [Integer] & 
        \( {TE \vdash i : integer} \) \\
    \end{longtable}
    \caption{}\label{s-empty}
\end{table} 

\begin{table}[H]
    \centering
    \begin{longtable}[c] { r c }
        [Double] & 
        \( {TE \vdash d : double} \) \\
    \end{longtable}
    \caption{}\label{s-empty}
\end{table} 


