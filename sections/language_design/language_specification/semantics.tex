\section{Semantics (and structural operational semantics figure) (EMPTY)}
The operational semantics are the rules that give our syntax meaning, and without them it would be impossible to make a working language, as a given syntax wouldn't have a clearly defined functionality. For this reason, we will define the operational semantics in this section.

There are 2 different ways of describing semantics of a language. Big-step semantics and small-step semantics. While they are equivalent, there are some advantages and disadvantages to both of them. The big-step semantics evaluates the whole expression from the start to the end expression. Small-step semantics may only evaluate part of the expression one step at a time. The disadvantage of small-step semantics is that it can be daunting to both write and read.

We have decided to make our operational semantics a small-step semantic, as a small-step semantic ruleset is easier to extend in the case of a new feature compared to a big-step semantic ruleset. This means that it would be possible to add parallelity, if we realize that we need it at a later date.

\subsection{Boolean Expressions}
\begin{table}[H]
    \centering
    \begin{longtable}[c] { r c }
    \begin{tabular}{@{}c@{}} 
    [ASS1] &
    \\
    \\
    \end{tabular}
  \begin{tabular}{@{}c@{}}   \( <x := a, sto, env_l> \Rightarrow (sto[l -> v], envl) \)  \\ \( env_v,sto \vdash n \Rightarrow v \)
  \\ \( env_v,sto \vdash n \Rightarrow v \)
  \end{tabular}
        
 \end{longtable}
    \caption{Small-step semantics for boolean assignment expressions}\label{tab:my_label}
\end{table}
        
\begin{table}[H]
    \centering
    \begin{longtable}[c] { r c }
        
        [IF-ELSE-TRUE] & \( \dfrac{\langle \texttt{if} \ b \ \texttt{then} \ S_1 \ \texttt{else} \ S_2,\ s \rangle \Rightarrow \langle S_1,\ s\rangle}{if\ s\vdash b \rightarrow_b tt} \) \\[4ex]
        
        [IF-ELSE-FALSE] & \( \dfrac{\langle \texttt{if} \ b \ \texttt{then} \ S_1 \ \texttt{else} \ S_2,\ s \rangle \Rightarrow \langle S_2,\ s \rangle}{if\ s\vdash b \rightarrow_b ff} \) \\[4ex]
        
        [IF-TRUE] & \( \dfrac{\langle \texttt{if} \ b \ \texttt{then} \ S_1,\ s \rangle \Rightarrow \langle S_1,\ s \rangle}{if \ s\vdash b \rightarrow_b tt} \) \\\\
        
        [IF-FALSE] & \( \dfrac{\langle \texttt{if} \ b \ \texttt \ S_1,\ s \rangle \Rightarrow s}{if\ s\vdash b \rightarrow_b ff} \) \\[4ex]
        
 \end{longtable}
    \caption{Small-step semantics for boolean if statement}\label{tab:my_label}
\end{table}
        
\begin{table}[H]
    \centering
    \begin{longtable}[c] { r c }
        
        [WHILE] & \( \langle \texttt{while(} \ b\texttt{) \{} S \texttt{\}},\ s \rangle \Rightarrow \langle \texttt{if} \ b \ \texttt{then} (S\texttt{;while (} b \texttt{) \{} S\texttt{\}}) \texttt{else skip,}\ s\rangle) \) \\[4ex]
        
 \end{longtable}
    \caption{Small-step semantics for boolean while statements}\label{tab:my_label}
\end{table}
        
\begin{table}[H]
    \centering
    \begin{longtable}[c] { r c }
    
        [END] & END\\
    \end{longtable}
    \caption{Small-step semantics for end}\label{tab:my_label}
\end{table}