\section{Return statement visitors}
In this section the return statement visitors is discussed.\\
Visitor is in plural, because there are multiple visitors in the Eziuno compiler, that specifically handles return statements.\\
Ezuino implements two visitors that performs checks related to return statements. We will be discussing them in order or mention in this section. The first is the Return statement type check visitor and the second in the Missing return statement visitor.

\subsection{Return statement type check visitor}
The Return statement type check visitor is implemented in the class \texttt{ReturnStmtTypeCheckVisitor} and has the purpose of verifying the type of return statements.\\
Since the only purpose of this visitor is to check the type of return statements, the visitor only works with the nodes in the AST, which has to do with return statements. This means nodes which can put the return statement into a nested scope and the \texttt{Return\_stmtNode}.
\\\\
When a function is declared the type, it was declared as temporarily saved in a symbol table. When a return statement is reached, the return statements type is compared with a type of the last function that was declared which was saved in the symbol table within that scope. This ensures that if there are several nested function declarations, the return type is compared to the most recent “active” scope.
\\\\
To check the type of the return statement, the visitor implements the same private function the type checker uses to compare types. The function compares the types of two \texttt{ITypeNode} nodes. It can be found in listing \ref{code:TC:checkType}.\\
The actual comparison check for the return statement nodes occurs during the visit of the \texttt{Return\_stmtNode}. Listing \ref{code:RSTC:return} shows this on line 4.
\begin{lstlisting}[caption={Visit return statement node in the return statement type check visitor}, label={code:RSTC:return}]
$$@Override
public void visit(Return_stmtNode node) {
    Func_defNode funcdefnode = (Func_defNode) symtable.retrieveSymbol(FUNC_DEF_ID);
    checkType(funcdefnode, node);
}
\end{lstlisting}
\noindent

\subsection{Missing return statement visitor}
The Missing return statement visitor is implemented in the \texttt{MissingReturnStmtVisitor} class and has the purpose of verifying that every function has a return statement, unless it's a void function.\\
The idea behind this that the Ezuino compiler follows the rule: "If a function is not a void function, it should only be legal if it is guaranteed that it will reach a return statement."\\
This means that if there is no return statement in the scope of the definition body, it must be because there is an if- and else block in the scope of the body where every if statement and the else statement have a return statement.\\
If function declaration breaks this rule, an error has occurred and is thrown to the errorHandler.
\\\\
The two listing below shows examples of how the Missing return statement visitor handles different nodes.\\
Listing \ref{code:MISRE:ifstmt} shows the visit for an if statement node.\\
If an return statement in an if else block is guaranteed, it is noted for the current scope.
\begin{lstlisting}[caption={Visit for a if statement node in the Missing return statement visitor.}, label={code:MISRE:ifstmt}]
$$@Override
public void visit(If_stmtNode node) {
    symtable.openScope();
    node.getIfBlock().accept(this);
    boolean ifBlockHasReturnStmt = blockNodeHasReturnStmt();
    symtable.closeScope();
    BlockNode elseblock = node.getElseBlock();
    
    if (elseblock != null) {
        symtable.openScope();
        elseblock.accept(this);
        boolean elseBlockHasReturnStmt = blockNodeHasReturnStmt();
        ITypeNode returnStmt = getBlockReturnStmtNode();
        symtable.closeScope();

        if (ifBlockHasReturnStmt && elseBlockHasReturnStmt) {
            symtable.enterSymbol(BLOCK_RETURN_STMT, returnStmt);
        }
    }
}
\end{lstlisting}
\noindent\newline
Listing \ref{code:MISRE:funcdef} shows the visit for an function declaration node.\\
If there is no return statement in the current scope, or an if else block where return is guaranteed throw an error.
\begin{lstlisting}[caption={Visit for a function declaration node in the Missing return statement visitor.}, label={code:MISRE:funcdef}]
$$@Override
public void visit(Func_defNode node) {
    symtable.openScope();
    node.getBlockNode().accept(this);
    if (node.getType() != Type.VOID && (!blockNodeHasReturnStmt())) {
        errorHandler.returnNotGuaranteed();
    }
    symtable.closeScope();
}
\end{lstlisting}
\noindent\newline