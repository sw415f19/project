\section{Code Generation}
Nearing the end of the compiler run, after having done all the contextual analysis described in the different visitor sections, you end up with a Decorated Syntax Tree (DST).\\
This tree has been decorated by all of the visitors previously described and therefore lives up to all rules of the programming language. This means that the DST contains all of the data, the other visitors has decorated the tree with. Because of this, we can now use the DST to generate code.\\
The purpose of this section, is to describe how the Eziuno compiler, uses the DST decorated by all other visitors, to generate code.
\\\\
There can be multiple code generators, as they follow the visitor pattern to generate the code. This is the point where the designer of the compiler decides what target language(s) the compiler should have.

\subsection{Target languages}\label{CodeGen:TargetLanguage}
There are several different target languages that would be wise to compile to. These could be C, Assembly or Java-bytecode. Considering that our language is intended for home automation purposes, compiling to C isn't a bad idea, as it makes it easier for us to maintain the compiler.\\
Furthermore, the gcc compiler is on many platforms and architectures, making it trivial for us to port it from one platform to another. Java bytecode is also a good option for a target language, as it runs on the JVM, which is implemented on many platforms too, though it should be noted, that java bytecode still works with a class-based structure, meaning we would have to model our imperative language in an object-oriented space.
\\\\
Compiling to an intermediate language is usually a decent idea, as it lets our own language build on top of an already existing compiler-collection, to allow our code to run on many platforms out of the box.\\
Considering that the alternative is to compile to a platform-specific assembly language and that there are better options for us to compile to, we will therefore not be making a platform-specific compiler.
\\\\
A good approach for high level languages is to translate to a intermediate low level language, that already have implementations for most hardware.\\
Implementing a language this way, means that you can implement a code generator for the high level language to target language, which then can be use to generate code for other languages.\\
Doing it this way, means that instead of making a high-level $*$ low-level number of compilers, you only have to make a high-level $+$ low-level number of compilers.\\
This does add abstraction, but might also add complexity if either the source language or the destination language, implements structures that are hard or impossible to implement.\\
Examples of compilers with intermediate languages are Java and C\#. Java compiles to Java Bytecode which is the instruction set of the Java Virtual Machine. C\# compiles to Common Intermediate Language, which is then run on the Common Language Runtime.

\subsection{Code generator visitors}
The Ezuino compiler implements three different code generator visitors, each their own target language. The three visitors are C code generator, Arduino code generator and lastly Java Bytecode code generator. The code generators will be discussed in this section, in the same order they were mentioned.
\\\\
As mentioned section \ref{CodeGen:TargetLanguage}, there are many options for different target languages for a compiler. For Ezuino, three targets have been chosen: Standard C, Arduino and Java Bytecode.\\
Since the goal of the project is to compile to Arduino, one might be confused as to why the Ezuino compiler also implements a C code generator. The C code generator was the first code generator to be implemented in the Ezuino compiler. It was used as an proof of concept code generator and was mainly used to easily verify and compile test programs, to a non-Arduino language platform.\\
Because Arduino is a language very similar to C, C was a perfect language to implement the first proof of concept code generator.\\
After the C code generator was fully implemented, because Arduino is so similar to C, the Arduino code generator was practically extended from the C code generator. Only minor changes to the original C code generator was necessary to convert the C code generation into Arduino code.\\
Additionally, the Arduino code generator implements code generation for special Arduino functions. these are only a part of the Arduino code generator, not any of the other code generators.
\\\\
%Next the Arduino code generator...
Because that the C code generator is so similar to the Arduino code generator, both of these will be discussed together.\\
The code generators all uses a Java StringBuilder to build up, what will eventually become the final program in the target language. Using the visitor pattern, the code generators visits each node in the DST and appends the necessary code to the StringBuilder. The code appended to the StringBuilder depends on which node is visited, when walking the DST.\\
The first code snippet is from the C code generator. This snippet shows how the necessary C headers are appended to the StringBuilder (listing \ref{io01}).\\
\begin{lstlisting}[caption={Visit start node in C code generator}, label={io01}]
$$@Override
public void visit(StartNode node) {
    // includes
    builder.append("#include <stdio.h>\n" +
        "#include <string.h>\n");
    // start tree search
    node.getDcls().accept(this);
    node.getStmts().accept(this);
    // prints the final string
    out.print(builder.toString());
}
\end{lstlisting}
\noindent\newline
The second code snippet is from the Arduino code generator. This shows two visit methods for special Arduino functions, in the Arduino code generator. The first is a method for generating a \texttt{Delay} method from Arduino. The second in a method for generating a \texttt{DigitalRead} method (listing \ref{io02}).
\begin{lstlisting}[caption={Visit delay node and digital read node in Arduino code generator}, label={io02}]
$$@Override
public void visit(DelayNode node) {
    builder.append("delay(");
    node.getParameters().get(0).accept(this);
    builder.append(");\n");
}

$$@Override
public void visit(DigitalReadNode node) {
    builder.append("digitalRead(");
    node.getParameters().get(0).accept(this);
    builder.append(")");
}
\end{lstlisting}
\noindent\newline
The last visitor is the Jasmin code generator.\\
Unlike the previous two visitors, the jasmin code generator doesn't compile to a C-like language, but rather to Jasmin, which is an assembler-representation of Java bytecode. This is quite different from parsing to C or Java, as the scoping rules are a bit different, either being completely flat inside a given function, or more of a nested structure if one is using classes and access modifiers.\\
This combined with a register-based storage model means, that the compiler has to manually keep track of the allocated registers, and when it is safe to overwrite a given register with another value. This is currently handled when the Jasmin visitor visits a block node in the AST.\\
The code snippet in question can be seen in listing \ref{code:JasminBlock}. Note that a copy of the variable environment is reset to its pre-block state after the entire block has been generated. This is so that a variable declared in an inner scope will take precedence when being referenced in an expression, compared to a variable of the same name declared previously in an outer scope, while still "deallocating" the inner variables after the block has been processed.
\begin{lstlisting}[caption={Code for visiting a block in the JasminCodeGeneratorVisitor}, label={code:JasminBlock}]
$$@Override
public void visit(BlockNode node) {
    List<String> oldVariableEnvironment = new ArrayList<String>(currentVariableEnvironment);
    List<String> oldLocalFunctions = new ArrayList<String>(currentLocalFunctions);
    // < Code for visiting children omitted for readability>
    currentVariableEnvironment = oldVariableEnvironment;
    currentLocalFunctions = oldLocalFunctions;
}
\end{lstlisting}
\noindent\newline
Furthermore, it should be considered that the JVM is a stack machine, and thus it should be taken into consideration, when writing code for it. Though this is mostly a fact to be considered when evaluating expressions, though it is also important when it comes to function calls (or method invocations, as they are parsed down to), as these use the top i elements of the stack as parameters, and then push the return value of the function back on the stack.\\
The stack also plays a role in some of the more complex implementations (namely prints and strings), as these use object references like in Java, but on the stack, like any other value. It isn't terribly hard to write the (java) method invocations, but you do have to keep your wits about you, as the order of the stack matters. The typing syntax of the JVM gets a bit longer and incomprehensible, when referencing custom types like "String".