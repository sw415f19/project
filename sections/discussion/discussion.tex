\chapter{Discussion}
In this chapter, we will discuss the aspects of the development of the Ezuino programming language. We will review the intended goal of the programming language and reflect whenever the final product reflects the goals, which were set for the programming language in the beginning
\\\\
In the problem definition, the following got defined:
\\\\
\textit{In order to answer an increasing demand for programming knowledge we will make an imperative programming language. A programming language which is easier to learn for people with little or no prior programming experience, in order for them to get knowledge in the field of programming.}
\\\\
The problem definition leans closely towards that the final programming language should be easily used for new or beginners. The syntax was designed within the language design criteria to be readable, writable and reliable. Using these criteria, we were able to choose a minimal syntax, allowing for the core functionality for an Arduino. Before starting to write the syntax, we are looking at the core programming language an Arduino is running, Arduino C which is C with Arduino libraries. We then took the main features from the Arduino C programming language and found multiple alternatives of which syntax is best for each feature. We decided to keep the language very minimal, so most data types and control flow functionality is limited to the users being able to learn it fast, and don’t get overwhelmed by the features available. In the beginning, we had a general idea of how the program syntax should look like, from when we as developers started to program. After implementing the syntax, we had a user test with a third-party test person, who fitted the end goal group well, as they did not have any previous programming experience. The feedback received by this was surprising well for most of the syntax, other than naming some attributes and strong typing. The test person managed to create a small program, using the help sheet provided, which is a very good early reflection towards the program definition, where it should be used or learned by the target group.
\\\\
During the development of the Ezuino programming language, reserved keywords were a challenge as well. The Java programming language, which is the language the compiler is in, has a lot of reserved keywords, which must be handled in the grammar files, and inside the compiler as well. As the end goal was towards the Arduino, which is using a library for most of its special syntax, the Ezuino programming language should know about these “special syntax” as well.
\\\\
Arduino uses the C programming language, with a special library designed to make development and usage of the Arduino easier for the users. 
The most basic of the functions in the library are implemented in Ezuino. While more should be added to make Ezuino better interact with the Arduino hardware, only a few were implemented as proof of concept.
\\\\
Scopes rules in the Ezuino programming language are static as it is the most common practice today, static typing was chosen to make programs more reliable and due to knowing more about these than dynamic typing. The block structure chosen is the nested scope-structure which are very similar to Arduino C, where only functions must be declared in the outer most scope was implemented. The advantage of nested scoping is that variables can be declared in each block which might be closer to the code where they are used making code more readable. Nesting functions in functions is also not allowed to prevent complexity.
\\\\
In the Ezuino programming language, variables are declared before doing statements. This is inspired from Pascal where they have a "special block" that makes it clear where variables are declared, although In Ezuino the special block is implemented in a way that requires less syntax. There has been some discussion after the user feedback whether it is good idea for our target group, therefore, it may be changed in the future if more user feedback can be assessed. Although new users with a little more experience in the language might find it normal, if it is the first programming language they have used.
 


