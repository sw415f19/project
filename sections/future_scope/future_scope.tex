\chapter{Future Scope of the Project}
If given more time to work on the project, it would make sense to improve several aspects of Ezuino. There were both feedback from the user we could account for and some features we think might make sense to extend the language with in the future.
The feedback from the user suggested that we should change our data type syntax and shift to weak typing. Data type names like text and float instead of string and double are more common terms for the data types that people without programming experience are more likely to understand. Weak typing also increases writability by making the language less strict, better accommodating people less experienced in using types in practice. This might lead to less reliability code due to using implicit casting without intending to. \\
We also got feedback on the dynamic array implementation called list, which we had intended to include in the language, but have not been able to due to lack of time. If given the time the data type should be implemented as it not only serves the purpose of arrays in Arduino C, which is essential for many programming applications, but also are an easy way to implement dynamic arrays. This data type would presumably have increased writability as dynamic arrays are fairly complex to implement in Arduino. This might also have increased writability due to new programmers more often opting to use dynamic arrays when it fits the situation more when a more high level data type is available. \\
We did not have time to implement the features determined to be “could have” in our Moscow. \\
The largest feature was the struct data type. Having the data type would increase readability and writability of Ezuino. Given enough development time, it would also make sense to increase the number of Arduino functions. At this point, the only functions Ezuino provides are the one we consider most basic and most essential. A parsing feature from string to other data is also something Ezuino currently lacking could be implemented. Finally if we wanted to make the language a bit more powerful and had fixed all the other issues, we could have implemented more assignment operators.

Besides the features we got feedback on and the features that was planned but not implemented, a more practical change that should be looked at would be to connect the Ezuino language directly with a compiler. Ezuino does not currently interact directly with the Arduino compiler or Arduino IDE. Code is generated either for the Arduino language or Java Bytecode. In order to actually compile the code the user would have to set the destination output to a file, that they can compile with a compiler that take either of these languages. This is not a particular user friendly and is of course something that would have to be solved if Arduino was to be published for general use. We did briefly researched two approaches that could be considered if Ezuino should be extended in the future.
\begin{itemize}
    \item Using a AVT library to compile c code and upload it directly to Arduino
    \item Using a embedded java environment to use Jasmin java binary classes to compile with HaikuVM\cite{haiku}, which would bring a micro-version of the JVM to the Arduino.
\end{itemize}
The options are unlimited when getting the code executed on the Arduino. This subject is to be researched further; we must assess if we need a custom coding environment.


%One way is to use a native implementation, which uses serial and parallel communication between a C compiler and the micro-controller on the Arduino. There are different AVR libraries which can compile C code and upload it to the Arduino and other microchips. It is also possible to use embedded java environment to use Jasmin java binary classes to compile with HaikuVM, which provides a solution to bring a micro-version of the JVM to the Arduino.\cite{haiku}\\
%The options are unlimited when getting the code executed on the Arduino. This subject is to be researched further; we must assess if we need a custom coding environment.

\iffalse
Besides those 



, being easier to was preferable as an additional option to arrays.

using the feedback from the user test; better type names, weak typing at least so int can be written in double types
dynamic arrays or as we called them list
\\\\
things we didnt generally have time for:

string concatenation in the Arduiono C code generator
records / structs are very useful for various tasks in arduino. For example to define data types related to whether when working with whether
parsing library. So string for example can be parsed to other data types.
\fi

