\section{Error Handling}\label{error-handling-chap}
We won't be discussing another visitor in the section, we will however be describing how the Ezuino compiler handles errors in general.\\
Error Handling is a vital general tool, which is used for the users to get an overview of the errors, which they made during their time coding the Ezuino programming language. The Ezuino error handling tool also implements a small part of the ANTLR Runtime library, which detects errors in the syntax.
\\\\
The Ezuino error handling system can be described in two parts. The first part is the part implementing the ANTLR Runtime libary. The second part is Ezuino's own custom error handling system.
\\\\
Ezuino uses the ANTLR libary to implement it's own version of the \texttt{ANTLRErrorListener} class. This is implemented in the class named: \texttt{ErrorListener}. The \texttt{ErrorListener} is used to check the data ANTLR generates, for syntax errors. If no syntax errors is found in the ANTLR generated data, the compiler continues.\\
However, if an error is found, it is added to a list of errors held within the \texttt{ErrorHandler} class. The \texttt{ErrorHandler} class is part of Ezuino's custom error system and will be described later.\\
Listing \ref{eh01} shows the implementation of the \texttt{ErrorListener} class and how it uses the \texttt{ErrorHandler} class to add a \texttt{syntaxError} to the handler class.\\
\begin{lstlisting}[caption={Start of the ErrorListener class}, label={eh01}]
private int errorCount;
private ErrorHandler errorHandler;

public ErrorListener(ErrorHandler errorHandler) {
    this.errorHandler = errorHandler;
    this.errorCount = 0;
}

$$@Override
public void syntaxError(Recognizer<?, ?> recognizer, Object offendingSymbol, int line, int charPositionInLine,
    String msg, RecognitionException e) {
    this.errorCount++;
    errorHandler.syntaxError(errorCount, msg, line, charPositionInLine);
}
\end{lstlisting}
\noindent\newline
Next we have Ezuino's own custom error handling system. As mentioned, the \texttt{ErrorHandler} class in part of this system. The \texttt{ErrorHandler} class also functions as the main class of Ezuino's error system. It's the class responsible for handling all errors that occurs in the compiler. The \texttt{ErrorHandler} (handler) holds a list of errors and a \texttt{errorCount} which it uses to keep track of all errors given to it. During compiling, Ezuino regularly checks that the handler has no errors. These checks occurs after each section, a section being either the syntax or contextual analysis. If one or more errors has occurs during one of the sections, the compiler will stop after the section and print the errors to the user.\\
Listing \ref{eh03} shows a code snippet of the \texttt{ErrorHandler}. On it, is the error message, error count and public method for printing all of the errors in the list.\\
\begin{lstlisting}[caption={Start of ErrorHandler class}, label={eh03}]
private List<ErrorMessage> messageList = new ArrayList<>();
private int errorCount = 0;

public boolean hasErrors() {
    return errorCount != 0;
}

public void printErrors(String reason) {
    if (hasErrors()) {
        StringBuilder sb = new StringBuilder();
        String errmsg = "##  " + reason + "\n -- ## ERROR OUTPUT CONSOLE ## -- ";
        sb.append(errmsg + "\n");
        for (ErrorMessage message : messageList) {
            sb.append(message + "\n");
        }
        System.err.println(sb.toString());
    }
}
\end{lstlisting}
\noindent\newline
Just like the \texttt{ErrorListener} used the handler class to add errors, the visitors of Ezuino can also use the handler class to add errors that might occur during their execution. Because the visitors can use handler class to report their errors, it's possible to divide the compiler into different sections, as described earlier. This allows the compiler to stop compiling early, if an error has occurred.
\\\\
The Ezuino error system also implements custom error messages, which is designed to give more detailed error messages, whenever an error occurs. The custom error messages are built within the class \texttt{ErrorMessage}. Three different types of errors exist within the Ezuino error system. The different types are: \texttt{GeneralError}, \texttt{SemanticError} and \texttt{SyntaxError}. \texttt{SyntaxError} was the type of error the \texttt{ErrorListener} gave to the handler when an error occurred.\\
Which type of error that occurs are specified within the visitors. The error can be one of the three types and depending on how critical the error is, it can either be a warning or error.\\
Only errors stop the compiler from finishing. If a program has a warning, it will still be compiled. However, the warning will still be printed to the user.\\
Listing \ref{eh04} shows three examples of different errors the visitors can use to give detailed error messages to the user. The types of errors on the listing is: Syntax, General and General error.
\begin{lstlisting}[caption={three different error messages from the ErrorHandler class}, label={eh04}]
public void syntaxError(int errorCount, String msg, int line, int charPositionInLine) {
    String errorMsg = String.format("#" + errorCount + " - " + "Error parsing expression: '%s' on line %s, position %s", msg, line, charPositionInLine);
    addError(new SyntaxError(ErrorType.ERROR, errorMsg));
}
public void funcAlreadyDeclared(String character) {
    addError(new GeneralError(ErrorType.ERROR, "Function " + character + " is already defined in this scope."));
}
public void unexpectedType(ITypeNode node, Type type) {
    addError(new GeneralError(ErrorType.ERROR, "Unexpeced type! Expected: " + type.name() + ", was " + node.getType().name() + " - Node: " + node));
}
\end{lstlisting}
\noindent\newline