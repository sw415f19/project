\section{Targeted group}
\label{targetedGroup}
Before we can start working on language design and ideas, a targeted user group must be found and analyzed, such that we can gather ideas and create a language, which resides towards the initial problem statement.  \\ 
\\ 
The target group in which this project would get focused on are the ordinary everyday individual who lives in a house or apartment. Since this may be a very abstract target group, the experience on how to use the product may be different. In such, a PACT analysis got made in order to understand the targeted group better. \\ 
\\ 
\textbf{People (P)}: The focused group is people of every class, who live in a home. This could be anyone, from larger families to people living by themselves. The age is not restrictive but should be focused to be learned by people who are 18 years or older. The learning experience on how to use technology vary since seniors can have little to no experience on how to use technology. The focus group is between 18-45 year olds, who reside with some interest in technology.\\ 
\\ 
\textbf{Activities (A)}: The product should be able to automate a different task, regarding electronics in general homes. For example when it is Spring, if it gets too warm in the house, the product should be able to automatically open the window and close the window again after it has registered the room temperature. The signal would be when the room temperature cold , and when the temperature is warmer than normally.  \\ 
\\ 
\textbf{Context (C)}: The product should be set up within an indoor and outdoor apartment, in all types of weather. The product can be used at all times of the day and require an Arduino to program the different devices.\\ 
\\ 
\textbf{Technologies (T)}: The technology will use Arduino as the hardware and a charger that will keep the Arduino running. The user should be able to access the code, by any electronic device capable of coding, and should be compatible with any device able to connect and run Arduino hardware. \\ 
\\ 
This PACT analysis will not be used to get an understanding of the programming paradigm of the code. As we reached the target group and what they must do, we should be able to narrow down the group. As we have narrowed them down, we can analyze the product, and start to get ideas about the syntax.