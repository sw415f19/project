\section{User Test}
In this chapter, we present the feedback we were given from Ezuino. The test person was a student in psychology second semester. The testperson has never programmed before but was interested in participating in the test. The test was conducted by giving the test person a guide that both covered basic programming terms, the syntax of Ezuino with code examples, and a set of assignments. The guide can be seen in appendix \ref{user_guide_appendix}. The test person was taking the test together with a test moderator, in order to ensure the participant did not get completely stuck at a task either due to being completely new to programming or due to the guide being unclear. After the test, the participant was asked to give feedback on the programming language and to review key points about Ezuino where it differed from C. The screen and communication between the test moderator and the test person was recorded for internal review at a later time.
We got the following feedback
\begin{itemize}
 \item “AND” and “OR” was preferable to $\&\&$ and $\|\|$ and boolean true/false was preferred to 1 and 0.
 \item The data types could be less technical named, so it is easier to understand without a background. An example from the test person was that string could be called “words”. We asked if “text” would also be easy to understand as we felt it was more precise for more use cases of strings, and the test person said it was just as good. It could be observed from the test person that they fairly fast forgot was the meaning of int, double and boolean was after having read the explanation. While this could simply be due to having to comprehend many programming terms in a short time span, it could also be due to the name of the data types being less intuitive, making them harder to remember the meaning of.
 \item Strong typing also came as a surprise to the test person. Attempts of assigning a double variable to an integer("32"), was an example of this.
 \item We showed the test person the list we had intended to implement as versions of arrays that did not need malloc and free and functioned like ArrayList in Java but without being object oriented. The test person did prefer it but also suggested a method to find the index of a value in a string list in case they had a hard time keeping track of what index each value was saved. Interestingly enough the index of a value is often easier to remember in arrays compared to list when looking at the source code since you have to use the array index when you assign entries in an array. This suggests that having both arrays and a structure similar to list would be good since they both have different advantages for people new to programming. In both arrays and list the test person would prefer that entries in the array would start from index 1 instead of 0.
 \item According to the test person “:=” and “=” made sense compared to C, “:=” reminding the test person of assignment in a math program previously used in her study.
 \item During the test there were a few times the test person forgot declaration before assignment. This could be interpreted as a declaration before usage being unintuitive to new programmers.
\end{itemize}

Besides those points the test person also gave some feedback on our current error messages, making us aware that both the error messages from type mismatch and syntax error had some room for improvement.
This feedback confirms that most of the syntax we differ from C makes sense from a user perspective of our target group. It also gives feedback to several ways we could change the syntax, and the type checking to make it better suit new programmers needs.