\section{Syntax Specification}
Now that we have been through the different thoughts about the parsing techniques and we can move on towards the actual syntax and programming example of the Ezuino programming language.
This section will give an overview of the syntax of the Ezuino programming language syntax, the reserved keywords and the context free grammar that is implemented by the scanner. At the end of the section there will be a few programming examples in Ezuino of programs with correct syntax.
\subsection{Keywords in the Ezuino Programming Language}
These are the keywords used in the Ezuino programming language, without the reserved and Arduino based syntax.
\begin{table}[H]
\begin{tabular}{|c|c|c|c|}
\hline
\textbf{Syntax Name} & \textbf{Syntax}      & \textbf{Syntax Name} & \textbf{Syntax}         \\ \hline
int                  & Integer declaration  & !                    & Not                     \\ \hline
double               & Double delcaration   & !=                   & Not Equal               \\ \hline
string               & String declaration   & \textless{}          & Less Than x             \\ \hline
boolean              & Boolean declaration  & \textless{}=         & Less Than or Equal x    \\ \hline
:=                   & Assign Statement     & \textgreater{}       & Greater Than x          \\ \hline
+                    & Plus Operator        & \textgreater{}=      & Greater Than or Equal x \\ \hline
-                    & Minus Operator       & if                   & If Statement            \\ \hline
/                    & Division Opeartor    & else                 & Else Statement          \\ \hline
*                    & Mult Operator        & while                & While Statement         \\ \hline
Loop                 & Loop Structure       & Setup                & Setup Structure         \\ \hline
func                 & Function Declaration & return               & Return Statement        \\ \hline
AND                  & Logical AND Operator & OR                   & Logical OR Operator     \\ \hline
=                    & Equal Operator       & :=                   & Assignment Operator    \\ \hline


\end{tabular}
\end{table}
\subsection{Reserved words \& Keywords}
In the Ezuino programming language, a reserved keyword is vital to prevent users to overwrite, define and use keywords, which is used in the programming language already. In case a user tries to define a variable or function named a reserved keyword, they will receive an error, which will be thrown into the error handler. The error handler will then tell the users, where and which reserved keyword they used where. The reserved keywords can be found in table \ref{reservedKeywordsList}.

\begin{table}[H]
\centering
\caption{Table of Reserved Keywords}
\begin{tabular}{llll}
if           & switch      & int         & boolean     \\
return       & AND         & OR          & true        \\
false        & int         & double      & main        \\
string       & while       & else        & Print       \\
DigitalWrite & DigitalRead & AnalogWrite & AnalogRead  \\
Delay        & DelayMicro  & PinMode     & SerialBegin \\
SerialEnd    & Setup       & Loop        & goto        \\
float        & for         & ArrayList   & List        \\
Collection   &             &             &            
\end{tabular}
\label{reservedKeywordsList}
\end{table}