\section{User Test}
In this chapter, we present the feedback we were given from Ezuino. The test person was a student in psychology second semester having never programmed before but interested in participating in the test. The test was conducted by giving the test person a guide that both covered basic programming terms, the syntax of Ezuino with code examples, and a set of assignments. The guide can be seen in appendix \ref{user_guide_appendix}. The test person was taking the test together with a test moderator, in order to ensure the participant did not get completely stuck at a task either due to being completely new to programming or due to the guide being unclear. After the test, the participant was asked to give feedback on the programming language and to review key points about Ezuino where it differed from C. The screen and communication during the test was recorded for intern review with consent.
We got the following feedback
\begin{itemize}
 \item “AND” and “OR” was preferable to $\&\&$ and $\|\|$ and boolean was preferred to 0 and 1.
 \item The data types could be less technical named, so it is easier to understand without a background. An example from the test person was that string could be called “words”. We asked if “text” would also be easy to understand as we felt it was more precise for more use cases of strings, and the test person said it was just as good. It could be observed from the test person that they fairly fast forgot was the meaning of int, double and boolean was after having read the explanation. While this could simply be due to having to comprehend many programming terms in a short time span, it could also be due to the name of the data types being less intuitive, making them harder to remember the meaning of.
 \item Strong typing also came as a surprise to the test person. That 32 gave a type error when assigned to double as an example of this.
 \item We showed the test person the list we had intended to implement as versions of arrays that did not need malloc and free and functioned like ArrayList in Java but without being object oriented. The test person did prefer it but also suggested a method to find the index of a value in a string list in case they had a hard time keeping track of what index each value was saved. Interestingly enough the index of a value is often easier to remember in arrays compared to list when looking at the source code since you have to use the array index when you assign entries in an array. This suggests that having both arrays and a structure similar to list would be good since they both have different advantages for people new to programming. In both arrays and list the test person would prefer that entries in the array would start from index 1 instead of 0.
 %\item To our surprise the test person did not dislike the idea of ending statements with “;” like in C, comparing it to dot in a sentence. According to the test person, dot could even substitute “;”. In the same manner of thought, the test person also suggested that the starting letter of keywords that can start a line could also start with uppercase, making it more like a sentence.
 \item According to the test person “:=” and “=” made sense compared to C, “:=” reminding the test person of assignment in a math program previously used in her study.
 \item During the test there were a few times the test person forgot declaration before assignment. This could be interpreted as a declaration before usage being unintuitive to new programmers.
\end{itemize}

Besides those points the test person also gave some feedback on our current error messages, making us aware that both the error messages from type mismatch and syntax error had some room for improvement.
This feedback confirms that most of the syntax we differ from C makes sense from a user perspective of our target group. It also gives feedback to several ways we could change the syntax, and the type checking to make it better suit new programmers needs.









%This guide can be seen in the appendix (SKRIV APPENDIX IND).  
%The test setup is simple. A table, with a chair together with a computer running a text editor with one text file “code.ezuino”, and running the programming language in the background. After the compiler has compiled the program, the output program was then copied into an Arduino IDE, and executed on the Arduino. The test person was handed a syntax specification table, which contains a small explanation of what each operator did, and we tried to give them two simple tasks, which they will use the language, only using the syntax specification table. \\
%\\
%The first test which got presented to the test person was a simple programming task, where they had to construct a “hello world” string in a function, which is outside the setup and loop function, and run it the function once. At first, the test person was confused, however, after presenting them to how the setup and loop functions work, and told them what each of them did, they replicated a function, similar to the setup function, and insert a print statement, which printed “hello world”. The test person had an issue understanding functions, as they wanted to run the program right after they have added their new function. When a dialogue with the test person, on how functions work, and that the code only executes the setup and main, and you must call the function in each, the test person completed this task.\\
%\\
%The second test which got presented to the test person, were a harder task, but builds on the previous task, and provides a code example written C, with Arduino libraries attached. The tasks were to learn about the user’s interaction between the Arduino and applying a code to get a result. The programming was to get data from a temperature sensor and get the temperature out in the Arduino console. By already providing the body, with setup, loop and serial begin as the first test setup, the only explanation we gave the test person beforehand, was that floats form the program example was double in our language, and that they had the same functionality. As the Ezuino programming language already had a lot of the library features which Arduino reside, the user could replicate the code into the Ezuino programming language, with one missing component, the language was missing (println – prints on a new line). We asked the test person whenever he understood the difference between the normal print and the print new line method, and he answered yes, and concluded that it could be a good feature to include.\\
%\\
%We can conclude from this limited test that language specification and documentation are everything. Good explanations of where to start, what each does, and where to use each feature where is vital, and can get people, with no programming experience to produce functional code. In goal with this project, is to avoid having a teacher, sitting next to the person programming, but to learn it from a set of instruction and specification. The test setup with the temperature sensor used for the second test was not user friendly for users who do not have experience at all. If home automation for Arduino would ship, we should provide easy and ready packages, where the user only should plug the Arduino into their computer and execute the code. At last, there was a feature which we did not think should have been added. The println feature was understood by the user, and as it can provide a good functionality to the program, we added this feature in the language.
%\\