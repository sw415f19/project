\section{Current devices and programming languages}

With the rise of cheap microprocessors and humans innovation came an abundance of new small internet connected devices. These devices are what is called the "Internet of Things". Gadgets, machines and appliances which are connected to the internet\cite{iot-definition}. A subgroup within the IoT-subject is Home Automation. Home automation is the process for consumers to automate their home. Some examples are garage doors, light bulbs, thermostats, smart locks, outlets and even curtains. In a research paper\cite{home-auto-growth} on the smart home market, it was clear that in 2017, the worldwide customer spending on smart home was \$84 billion and projected to grow to \$155 billion by 2023. For the hometinkers and hobby-hackers, these products were available too. It started with the Arduino in 2003 which had hardware that was cheap and easy to program\cite{arduino-history}. It wasn't until later that the Arduino gained WiFi capabilities. Pair the internet capabilities with the fact that the easy to learn Arduino C programming could be used to programming the chips and help from the online community and new ideas form. 

In the later years, the Arduino sparked a new flavour of Arduino like microcontrollers with both WiFi and bluetooth capabilities. Some examples of these internet connected microcontrollers are the ESP8266 (2014) and the ESP32 (2016). Both of which are as cheap as the Arduino (sometimes even \$1), available from most of the world in a GPIO pins that resembles the Arduino, with the added benifit of having WiFi and bluetooth connectivity. However, the programming language on these are different. Since Arduino C is so popular, a community effort has been made to add support for the Arduino SDK to these chips. That means that multiple of the IoT devices are supported by the Arduino IDE and can be written in Arduino C, with the proper libraries.

From a business standpoint, many have started making their products IoT-devices, most of which can be controlled directly from your phone. It opened up new possibilities for doing stuff without having to do them manually. Since companies started creating all the IoT-devices, companies like Google and Apple have begun up for integrating with the home automation devices to lure users onto their platform. Google' Assistant\cite{google-assistant}, Apple's Home\cite{apple-homekit} and Amazon's Alexa\cite{amazon-alexa} are currently the major players in this. They all allow the user to control their automated home in a more centralised and streamlined way than having to control each device from their own app. May it simply be lights to turn on (Eg. Ikea Trådfri bulbs), temperature regulation from night to day (ecobee thermostat) to smart connected fridges (Eg. Samsung Family Hub fridges) amongst other things. 