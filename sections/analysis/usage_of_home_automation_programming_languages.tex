\section{Usage of home automation programming languages} % find ekstra på hvilke sprog der bliver brugt til de nuværende automatiserings systemer.
In regards to current home automation, it is possible to divide the existing systems into three different generations\cite{chinese-home-automation}:
\newline
The first generation home automation is a system in which home appliances communicate with a proxy server, which acts as a centralized control unit for the user to interface with. An example of a generation 1 system is ZigBee\cite{zigbee}, which essentially is a collection of modules communicating to a receiver though a wireless network. The user can then program the connected devices through an interface (usually a phone or a desktop/laptop) to achieve the desired behaviour.
\newline
The second generation home automation is a system in which an AI manages home appliances, rather than relying solely on human programming. The most notable examples of this is Google Assistant or Amazon Echo, which strictly speaking are "only" virtual assistants, though also very powerful when interfaced with home appliances. Unlike a first-generation system, these systems learn from the users to better suit their needs. The most simple way to describe a second generation system is to consider it a first generation system with some form of AI layered on top of it.
\newline
The third generation of home automation is characterized by robots who operate semi-autonomously, if not completely without human input. An example of this would be a Roomba, that vacuums the floors in a house without much human interaction required.
\newline