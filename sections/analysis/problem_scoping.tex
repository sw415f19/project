\section{Problem scoping}
This section will be scoping the problem and the product onwards based on the MoSCoW analysis, and the decisions described in this section.  \\
\\
The programming language will create a full code generation, which will be mainly focused and limited to Arduino primary programming language, C. The project will only be fully implemented in this language; however, some Java Bytecode generation will be made as well, but not to the full extent. As there is a lot of IoT devices, and other Arduino models, the project will be limited to be developed and tested on an Arduino Uno.  \\
Syntax of the programming language will be limited to the analysis part, which will take part in the most common syntax used in programming languages today. The functionality and syntax of the programming language will only be limited to the most crucial parts mentioned in the MoSCoW analysis. All the additional functionality, like Java’s Array List, will not be within this project scope at this time but will be available for further development. \\
As it is essential that the compiler does not contain errors, testing will get focused to a greater extent of this project. Testing of the program, JUnit test should be done in the significant fields of code and functionality, together with running automated Continuous Integration (CI), which will run the tests in every build, together with a PR/Code Quality manager running as a static code analyser. Further testing, such as code coverage test will not get considered in this development phase.  \\


As the scoping reach an end, we can start putting a problem definition together, where we can define the exact problem that we in which can solve with this programming language.  